
\chapter{Python language syntax}
\label{chap:python}

This chapter features a short introduction to the syntax of the
programming language Python, and an explanation of the commonly-used
constructs and idioms that may appear obscure.  This is by no means a
complete overview of the syntax, nor a formal definition of Python;
the \href{http://docs.python.org/2.6/}{official Python documentation}
\cite{python:docs, python:reference} should be consulted for this
purpose.

According to the
\href{http://en.wikipedia.org/wiki/Python_(programming_language)}
{Wikipedia article on Python}, ``Python is a general-purpose
high-level programming language. Its design philosophy emphasizes code
readability. Python claims to `\emph{[combine]} remarkable power with
very clear syntax'. It has a relatively uncluttered visual layout,
uses English keywords frequently where other languages use
punctuation, and has notably fewer syntactic constructions than other
popular structured languages \emph{[...]}\'''.


\section{Basics of Python language syntax}
\label{sec:syntax}

Python is an \emph{imperative} programming language, meaning that
every Python program is a sequence of statements to be executed by the
computer.  Each statement is terminated by a newline character, but a
statement is automatically continued on the following line if there
are imbalanced parentheses.  In other words, a statement ends at the
first newline character such that all parentheses are balanced, that
is, every occurrence of `\verb"("', `\verb"["', or `\verb"{"' has a
matching occurence of the closing symbol.
\begin{lstlisting}
# single-line statement
print (21*2)

# the following statement is automatically continued
# until the newline following the last closing parenthesis
u = (1 + 4*x[2] 
       + 2*y[1 + n[42] ]
       + z)
\end{lstlisting}
As one can see from the example above, the Python interpreter ignores
any character following `\verb"#"' until the end of the line.  This
syntax is used to add explicatory comments to the Python source code.
The Python interpreter completely ignores comments and blank lines
while reading a source file; that is, a Python program will have
\emph{exactly} the same behavior if all the comments and blank lines
are removed.

A distinctive feature of Python is its use of whitespace to delimit
program blocks: Python mandates that statements belonging in the same
sequence are aligned with the same amount of whitespace from column 0.
The indentation must increase in the block after a flow-control or
function definition.\footnote{Python syntax also mandates that a colon
  `\lstinline{:}' is the last non-whitespace character before a new
  block begins, so an effective rule of thumb is: if a line ends with
  a colon, the following one must increase indentation.}
A block ends upon encountering a statement with a lesser indentation
level (i.e., beginning closer to column 0).

For example, consider the following simple code:
\begin{lstlisting}
for x in range(10):
  # compute the square of x
  y = x*x
  # print it
  print y
# the following statement dedents, thus ending
# the body block of the "for" loop.
print "All done."
\end{lstlisting}
The assignment to variable `\lstinline{y}' and the `\lstinline{print}'
statement belong in the same code block (they body of the
`\lstinline{for}' cycle), therefore both begin in column 2.  The
`\lstinline{for}' and the last `\lstinline{print}' statement belong in
the same sequence and therefore start at column 0. 
                                                                          
                                                               

\subsection{Flow-control constructs}
\label{sec:flow-control}

Python supports the usual flow-control constructs
\lstinline{if}/\lstinline{else}, \lstinline{for}, \lstinline{while}.

The \lstinline{if}/\lstinline{else} statements are used to select which of two
blocks of code will be executed.  The \lstinline{else} part is optional and
can be omitted.  Cascaded \lstinline{if}/\lstinline{else} occurrences (to
select the execution of a particular block of instructions) are best
written using the \lstinline{elif} keyword as a shorthand for 
`\verb"else: if:"'. 

The \lstinline{while} statement repeatedly executes the ensuing code block;
the expression specified after the `\lstinline{while}' keyword is evaluated
at the beginning of each execution cycle, if the result is boolean
false, then execution of the code block is aborted and the program
flows continues after the block following the \lstinline{while} statement.
Therefore, if an expression evaluates to false at the first entrance
in the \lstinline{while} block, the block is never executed.

The \lstinline{for} statement repeatedly executes the ensuing code block,
each time binding the specified variables to a new value of the
sequence following the `\lstinline{in}' keyword.
\begin{lstlisting}
for x in range(10):
  print(x)
\end{lstlisting}
The `\lstinline{range}' function is provided as a convenience to specify
numerical integer sequences; by Python convention, the range is always
half-open: \verb"range(10)" is the sequence of natural numbers $\{0,
1, \ldots, 9\}$.  More formally, we have:
\begin{align*}
  \text{\lstinline{range}($n$,$m$)} &:= \{ x \in \setN : n \leq x < m \},
  \\
  \text{\lstinline{range}($n$)} &:= \text{range(0,$n$)}
\end{align*}

\subsubsection{Exception handling: \lstinline{try}/\lstinline{except}/\lstinline{finally} statements}
\label{sec:try-except-finally}

Exceptions 

\subsubsection{The `{\lstinline{pass}}' statement}
\label{sec:pass}

Python syntax does not allow to specify an empty block of code. There
is thus a special \lstinline{pass} statement that performs no
computational operation, and is used to define ``no-op'' blocks.

The main use of \lstinline{pass} is in combination with an
\lstinline{except} statement, to ignore a particular error condition:
\begin{lstlisting}
try:
  print (a[42])
except KeyError:
  pass # ignore error if "a" has no item "42"
\end{lstlisting}

% function definitions
% - docstrings
% - nested "def" / lexical closures



% - "%" formatting
% - list comprehensions / kludges for dict,set types
% - decorators
% - comments and docstrings


%%% Local Variables: 
%%% mode: latex
%%% TeX-master: "index"
%%% End: 
