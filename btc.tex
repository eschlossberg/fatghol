%auto-ignore


\chapter{Braided Tensor Categories, etc.}
\label{cha:btc}

This chapter summarizes the definition and properties of those
enriched category structures that underlie the graphical notation.
These axioms emerged parallelly with the development of the graphical
calculus, as different categories of graphs were investigated
\cite{turaev;yang-baxter}, \cite{freyd-yetter;btc},
\cite{joyal-street;tensor-calculus}, \cite{shum;tortile-categories}.
The reader is referred to \cite{joyal-street;btc} for a comprehensive
treatment, details, and proofs.

As terminology in these matters is not yet generally agreed upon, I
will adhere to nomenclature used by Joyal and Street in
\cite{joyal-street;tensor-calculus, joyal-street;btc}. 


\section{Monoidal Categories}
\label{sec:monoidal-categories}

Recall that a monoidal category $\A = (\A, \otimes, I, a, l, r)$ is a
category equipped with a ``tensor product'' bifunctor $\otimes: \A \times \A \to
\A$ which:
\begin{enumerate}
\item is associative up to a natural isomorphism $a = a_{ABC}:
  (A\otimes B)\otimes C \to A\otimes(B\otimes C)$ satisfying the ``pentagon condition for
  associativity'': 
  \begin{equation*}
    \xymatrix{%
      \bigl((A\otimes B)\otimes C\bigr)\otimes D
      &
      (A\otimes B) \otimes (C\otimes D)
      &
      A \otimes \bigl(B\otimes(C\otimes D)\bigr)
      \\
      \bigl(A\otimes(B\otimes C)\bigr) \otimes D
      &
      &
      A\otimes\bigl((B\otimes C)\otimes D\bigr)
      \ar "1,1";"1,2"^{a}
      \ar "1,2";"1,3"^{a}
      \ar "1,1";"2,1"_{a \otimes \id}
      \ar "2,1";"2,3"_{a}
      \ar "2,3";"1,3"_{\id \otimes a}
      }
  \end{equation*}
\item has a unit $I \in \A$ again up to a natural isomorphism, that is,
  \begin{equation*}
    l = l_A: I \otimes A \to A, \qquad r = r_A: A \otimes I \to A,
  \end{equation*}
  are natural isomorphisms constrained by the ``triangle condition for
  identity'':
  \begin{equation*}
    \xymatrix{%
      (A \otimes I) \otimes B
      &
      &
      A \otimes (I \otimes B)
      \\
      &
      A \otimes B
      &
      \ar "1,1";"1,3"^{a}
      \ar "1,1";"2,2"_{r \otimes \id}
      \ar "1,3";"2,2"^{\id \otimes l}
      }
  \end{equation*}
\end{enumerate}

\begin{proposition}[{\cite[prop.\ 1.1]{joyal-street;tensor-calculus}}]
  In any monoidal category the following diagrams commute:
  \begin{equation*}
    \xymatrix{%
      (A\otimes B)\otimes I
      &
      &
      A\otimes(B\otimes I)
      \\
      &
      A\otimes B
      &
      \ar "1,1";"1,3"^{a}
      \ar "1,1";"2,2"_{r_{A\otimes B}}
      \ar "1,3";"2,2"^{\id \otimes r_B}
      }      
    \qquad
    \xymatrix{%
      (I\otimes A)\otimes B
      &
      &
      I\otimes(A\otimes B)
      \\
      &
      A\otimes B
      &
      \ar "1,1";"1,3"^{a}
      \ar "1,1";"2,2"_{l_A \otimes \id}
      \ar "1,3";"2,2"^{l_{A\otimes B}}
      }
  \end{equation*}
  Moreover, $l_I = r_I$.
\end{proposition}

A monoidal category such that $a$, $l$, $r$ are the identity
transformations is called \emph{strict}. A \emph{tensor category} is
an abelian monoidal category such that the bifunctor $\otimes$ is bilinear.

\begin{example}
  The category $R\text{-}\catMod$ of (finitely generated) bimodules
  over a ring $R$ provides a trivial example of a monoidal category,
  when equipped with the usual tensor product and the obvious
  isomorphisms as $a$, $l$, $r$.
\end{example}

\begin{example}[Super Vector Spaces, cf.\ {\cite[example
    2.2]{joyal-street;btc}}] The category $\catVect^{\setZ_2}$ of
  $\setZ_2$-graded vector spaces has a monoidal structure with
  non-trivial associativity constraint: for $U, V, W \in
  \catVect^{\setZ_2}$ and homogeneous elements $u \in U$, $v \in V$, $w \in
  W$, we set
  \begin{equation*}
    a_{UVW} \bigl( (u \otimes v) \otimes w \bigr) = (-1)^{uvw} u \otimes (v\otimes w),
  \end{equation*}
  that is, we change sign if $u$, $v$ and $w$ are all odd.
\end{example}

Let $\A$, $\B$ be monoidal categories. A monoidal functor $F = (F,
\digamma_0, \digamma_2)$ consists of a functor $F: \A \to \B$ together with natural
transformations
\begin{align*}
  \digamma_2 = F_{2, AB} &: FA \otimes FB \to F(A\otimes B),
  \\
  \digamma_0 &: I_\B \to FI_\A,
\end{align*}
such that the following diagrams commute:
\begin{equation*}
  \xymatrix{%
    FA \otimes FB \otimes FC
    &
    F(A\otimes B) \otimes FC
    \\
    FA \otimes F(B\otimes C)
    &
    F(A\otimes B\otimes C)
    \ar "1,1";"1,2" ^{\digamma_2 \otimes \id}
    \ar "2,1";"2,2" _{\digamma_2}
    \ar "1,1";"2,1" _{\id \otimes \digamma_2}
    \ar "1,2";"2,2" ^{\digamma_2}
    }
  \quad
  \xymatrix{%
    I \otimes FA
    &
    FA
    \\
    FI \otimes A
    &
    F(I\otimes A)
    \ar "1,1";"1,2" ^{l_\B}
    \ar "2,1";"2,2" _{\digamma_2}
    \ar "1,1";"2,1" _{\digamma_0 \otimes \id}
    \ar "2,2";"1,2" _{F(l_\A)}
    }
  \quad
    \xymatrix{%
    FA \otimes I
    &
    FA
    \\
    A \otimes FI
    &
    F(A\otimes I)
    \ar "1,1";"1,2" ^{r_\B}
    \ar "2,1";"2,2" _{\digamma_2}
    \ar "1,1";"2,1" _{\id \otimes \digamma_0}
    \ar "2,2";"1,2" _{F(r_\A)}
    }
\end{equation*}

Given a monoidal functor $F$, one can define natural transformations
\cite{joyal-street;btc} 
\begin{equation*}
  \digamma_n : FA_1 \otimes \cdots \otimes FA_n \to F(A_1 \otimes \cdots \otimes A_n)
\end{equation*}
by an inductive process: $\digamma_0$ is given, $\digamma_1 := \id$, $\digamma_2$ is
given, and $\digamma_{n+1}$ is the composite
\begin{equation*}
    FA_1 \otimes \cdots \otimes FA_{n+1}
    \xrightarrow{\digamma_n \otimes \id}
    F(A_1 \otimes \cdots \otimes A_n) \otimes FA_{n+1} 
    \stackrel{\digamma_2}\longrightarrow
    F(A_1 \otimes \cdots A_n \otimes A_{n+1}).
\end{equation*}
Then, the following triangle diagram commutes:
\begin{equation*}
  \xymatrix@C0pt{%
    FA_1 \otimes \cdots FA_m \otimes FB_1 \otimes \cdots FB_n 
    &
    &
    F(A_1 \otimes \cdots \otimes A_m \otimes B_1 \otimes \cdots \otimes Bn)
    \\
    &
    F(A_1 \otimes \cdots \otimes A_m) \otimes F(B_1 \otimes \cdots \otimes B_n)
    &
    \ar "1,1";"1,3" ^{\digamma_{m+n}}
    \ar "1,1";"2,2" _{\digamma_m \otimes \digamma_n}
    \ar "2,2";"1,3" _{\digamma_2}
    }
\end{equation*}

A monoidal functor is called \emph{strong} iff all $\digamma_n$ are
isomorphisms; in the sequel, all monoidal functors will be assumed to
be strong, unless otherwise noted. A monoidal functor is \emph{strict}
iff all $\digamma_n$ are the identity maps. For a monoidal functor to be
strong (resp.\ strict), it suffices that $\digamma_0$ and $\digamma_2$ are
isomorphisms (resp.\ identities).

\begin{theorem*}[{\cite[Thm.\ XI.3.1]{maclane;cwn}}]
  Any monoidal category $\A$ is equivalent to a strict monoidal
  category $\A^\Box$ through a pair of strong monoidal functors $F: \A
  \to \A^\Box$ and $G: \A^\Box \to \A$.  
\end{theorem*}
This theorem has two important consequences:
\begin{inparaenum}
\item
  we can replace any diagram of monoidal categories and monoidal
  functors by an equivalent diagram of strict categories and strict
  functors, and 
\item any diagram built using only the structure maps $\id$, $a$, $l$,
  $r$, of any monoidal category is commutative.
\end{inparaenum}


\subsection{Braidings on monoidal categories}
\label{sec:braidings}

A brading is one sort of a ``relaxed commutativity constraint''. We
quote the following definition from \cite{joyal-street;btc}.
\begin{definition}
  A braiding for a monoidal category $\A$ consists of a natural family
  of ismorphisms
  \begin{equation*}
    \tau = \tau_{AB}: A\otimes B \to B\otimes A
  \end{equation*}
  such that the two hexagonal diagrams \eqref{eq:B1} and~\eqref{eq:B2}
  commute:
  \begin{align}
    \label{eq:B1}\tag{B1}
    \xymatrix{%
      &
      (B\otimes A)\otimes C
      &
      B\otimes(A\otimes C)
      &
      \\
      (A\otimes B)\otimes C
      &
      &
      &
      B\otimes(C\otimes A)
      \\
      &
      A\otimes(B\otimes C)
      &
      (B\otimes C)\otimes A
      &
      \ar "2,1";"1,2" ^{\tau_{AB} \otimes C}
      \ar "1,2";"1,3" ^{a}
      \ar "1,3";"2,4" ^{B \otimes \tau_{AC}}
      \ar "2,1";"3,2" _{a}
      \ar "3,2";"3,3" _{\tau_{A,B\otimes C}}
      \ar "3,3";"2,4" _{a}
      }
    \\
    \label{eq:B2}\tag{B2}
    \xymatrix{%
      &
      A\otimes(C\otimes B)
      &
      (A\otimes C)\otimes B
      &
      \\
      A\otimes(B\otimes C)
      &
      &
      &
      (C\otimes A)\otimes B
      \\
      &
      (A\otimes B)\otimes C
      &
      C\otimes(A\otimes B)
      &
      \ar "2,1";"1,2" ^{A \otimes \tau_{BC}}
      \ar "1,2";"1,3" ^{a^{-1}}
      \ar "1,3";"2,4" ^{\tau_{AC} \otimes B}
      \ar "2,1";"3,2" _{a^{-1}}
      \ar "3,2";"3,3" _{\tau_{A\otimes B,C}}
      \ar "3,3";"2,4" _{a^{-1}}
      }
  \end{align}
  A tensor functor $F: \A \to \B$ is braided iff the following square
  commutes:
  \begin{equation*}
    \xymatrix{%
      FA\otimes FB
      &
      F(A\otimes B)
      \\
      FB\otimes FA
      &
      F(B\otimes A)
      \ar "1,1";"1,2" ^{\digamma_2}
      \ar "1,1";"2,1" _{\tau}
      \ar "1,2";"2,2" ^{F\tau}
      \ar "2,1";"2,2" _{\digamma_2}
      }
  \end{equation*}
\end{definition}
A braided monoidal category is a monoidal category together with a
chosen braiding $\tau$.

\begin{proposition}[{\cite[Prop.\ 2.1]{joyal-street;btc}}]
  In any braided monoidal category the following diagrams commute: 
  \begin{eqnarray*}
    \xymatrix{%
      A\otimes I
      &
      &
      I\otimes A
      \\
      &
      A
      &
      \ar "1,1";"1,3" ^{\tau_{AI}}
      \ar "1,1";"2,2" _{r_A}
      \ar "1,3";"2,2" _{l_A}
      }
    &
    \xymatrix{%
      A\otimes I
      &
      &
      I\otimes A
      \\
      &
      A
      &
      \ar "1,1";"1,3" ^{\tau_{IA}}
      \ar "1,1";"2,2" _{l_A}
      \ar "1,3";"2,2" _{r_A}
      }
    &
    \forall A \in \A.
  \end{eqnarray*}
\end{proposition}

Braided categories get their name from the usual group of braids,
which provides the first example of a braided monoidal category.
\begin{example}\label{xmp:braids}
  Let $B_n$ be Artin's group
  of braids on $n$ strands; the category $\catBraid$ has all braids in
  $\bigcup_n B_n$ as morphisms, with the proviso that two braids  $a$ and
  $b$ compose iff they belong to the same $B_n$, i.e., they are formed
  by the same number of strings. From the arrows-only description of a
  category (\csref{cha:arrows}), we can compute that the object
  set of $\catBraid$ is the set of natural numbers $\setN$, and that 
  \begin{equation*}
    \catBraid(n,m) = \emptyset \text{ if $n\neq m$,} 
    \quad 
    \catBraid(n,n) = B_n.
  \end{equation*}
  Thus every arrow is invertible, so $\catBraid$ is indeed a
  groupoid. The tensor product $m\otimes n := m+n$ makes $\catBraid$ into a
  strict monoidal category, which is braided by the family of
  isomorphisms
  \begin{equation*}
    \tau_{m,n}: m+n \to m+n, 
    \qquad
    \tau_{m,n} := {\xy*!C\xybox{%
      (0,0);(4,2)**\dir{-},
      (2,0);(6,2)**\dir{-},
      (1,0)*{\ldots},(5,2)*{\ldots},
      (1,-0.5)*\txt{$m$ strands},
      %% FIXME: passare sopra/sotto
      (4,0);(0,2)**\dir{-},
      (6,0);(2,2)**\dir{-},
      (5,0)*{\ldots}, (1,2)*{\ldots},
      (5,-0.5)*\txt{$n$ strands}
      }\endxy}
  \end{equation*}
\end{example}

The above example can be extended by considering colored braids; in
particular, if we take the colors to be the morphisms of a monoidal
category $\A$, then we obtain a braided strict monoidal category
$\A-\catBraid$, which has certain universality properties and is the
first and simplest instance of graphical calculus (see
\cite{joyal-street;btc}).

\begin{proposition}[{\cite[Prop.\ 2.1]{joyal-street;btc}}]
  In any braided monoidal category, the following diagram commutes:
  \begin{equation*}
    \xymatrix{%
      (A\otimes B)\otimes C
      &
      A\otimes(B\otimes C)
      &
      A\otimes(C\otimes B)
      &
      (A\otimes C)\otimes B
      \\
      (B\otimes A)\otimes C
      &
      &
      &
      (C\otimes A)\otimes B
      \\
      B\otimes(A\otimes C)
      &
      &
      &
      C\otimes(A\otimes B)
      \\
      B\otimes(C\otimes A)
      &
      (B\otimes C)\otimes A
      &
      (C\otimes B)\otimes A
      &
      C\otimes(B\otimes A)
      \ar"1,1";"1,2" ^{a} 
      \ar"1,2";"1,3" ^{\id\otimes\tau}
      \ar"1,3";"1,4" ^{a\inv}
      \ar"1,1";"2,1" _{\tau\otimes\id}
      \ar"1,4";"2,4" ^{\tau\otimes\id}
      \ar"2,1";"3,1" _{a}
      \ar"2,4";"3,4" ^{a}
      \ar"3,1";"4,1" _{\id\otimes\tau}
      \ar"3,4";"4,4" ^{\id\otimes\tau}
      \ar"4,1";"4,2" _{a\inv}
      \ar"4,2";"4,3" _{\tau\otimes\id}
      \ar"4,3";"4,4" _{a}
      }
  \end{equation*}
\end{proposition}
The preceding proposition can be reinterpreted in the graphical
notation as stating for the $\tau_{AB}$'s the analogues of the relations
between generators of the braid group on $3$ strings.

Let $\A = (\A, \otimes, I, a, l, r)$ be a monoidal category; define a
monoidal category $\A\rev = (\A, \boxtimes, I, a, l, r)$ by means of
$A \boxtimes B := B\otimes A$.  When $\A$ is equipped with a braiding $\tau$,
then
\begin{equation*}
  \tau\rev_{AB}: A\boxtimes B := B\otimes A \stackrel{\tau_{BA}}\to A\otimes B =:
  B\boxtimes A
\end{equation*}
is a braiding on $\A\rev$.

\begin{proposition}[{\cite[Example 2.5]{joyal-street;btc}}]
  \label{thm:rev}
  The assignment $F := \Id$, $\digamma_2 := \tau$, $\digamma_0 := \id$ defines a a
  braided monoidal isomorphism between $\A$ and $\A\rev$.
\end{proposition}

If $\tau_{AB}$ is a braiding on $\A$, then $\tau'_{AB} := (\tau_{AB})^{-1}$
is again a braiding on $\A$.
\begin{definition}
  If $\tau_{AB} \circ \tau_{BA} = \id_{AB}$ then the braiding is called a
  \emph{symmetry}.   
\end{definition}

We shall only be concerned with categories of vector spaces or
differential modules, and all of them will be symmetric.
\begin{example}
  \label{xmp:vec1}
  The category of vector spaces and linear maps is braided symmetric
  with the family of isomorphisms
  \begin{equation*}
    \tau_{AB}: A\otimes B \ni a\otimes b \mapsto b\otimes a \in B\otimes A.
  \end{equation*}
  More generally, the category of $\setZ$-graded vector spaces can be
  made into a symmetric monoidal one with the usual tensor product and
  the commutativity constraints
  \begin{equation*}
    \tau_{AB}: A\otimes B \ni a\otimes b \mapsto (-1)^{ab}b\otimes a \in B\otimes A.
  \end{equation*}
  Here and in the sequel, elements of a graded object appearing as
  exponents to a number will stand for their degree, i.e., \((-1)^a =
  (-1)^{\deg a}\); therefore, \((-1)^{ab} \not= (-1)^{\deg (ab)} =
  (-1)^{\deg a + \deg b} = (-1)^{a+b}\).\FIXME{Ripetizione di una
    convenzione gi\`a espressa nel primo capitolo.}
\end{example}
\FIXME{Non metto esempi di una categoria intrecciata ma non simmetrica
  perch{\'e} non servono nel seguito\ldots}


\subsection{Duality in monoidal categories}
\label{sec:duality}

Let $\A$ be a monoidal category, and $A,B \in \A$.
\begin{definition}
  The object $A$ is left dual to $B$ (resp.\ $B$ is right dual to $A$)
  iff there exist morphisms $\ev: A\otimes B \to I$, $\coev: I \to B\otimes A$ such
  that the following diagrams commute:
  \begin{equation*}
    \xymatrix{%
      A
      &
      &
      A\otimes B\otimes A
      \\
      &
      A
      &
      \ar "1,1";"1,3" ^{A\otimes\coev}
      \ar "1,1";"2,2" _{\id_A}
      \ar "1,3";"2,2" ^{\ev\otimes A}
      }
    \qquad
    \xymatrix{%
      B
      &
      &
      B\otimes A\otimes B
      \\
      &
      B
      &
      \ar "1,1";"1,3" ^{\coev\otimes B}
      \ar "1,1";"2,2" _{\id_B}
      \ar "1,3";"2,2" ^{B\otimes\ev}
      }
  \end{equation*}
\end{definition}
One can show that $A$ and $B$ are dual to each other through $\ev$ and
$\coev$ iff the maps
\begin{align*}
  \ev^\sharp   &: \A(X,B\otimes Y) \ni f \mapsto (\ev\otimes Y)\circ(A\otimes f) \in \A(A\otimes X, Y),
  \\
  \coev^\flat &: \A(A\otimes X,Y) \ni g \mapsto (B\otimes g)\circ(\coev\otimes X) \in \A(X, B\otimes Y),
\end{align*}
are bijections, that is, $(\ev, \coev)$ is an adjunction pair for the
functors $A\otimes-$ and $B\otimes-$.

\begin{proposition}[{\cite[p.\ 71]{joyal-street;btc}}]
  \label{thm:funct-dual}
  Monoidal functors preserve duality relations.  
\end{proposition}
\begin{proof}
  If $F: \A \to \B$ is a monoidal functor, and $A$ is left dual to $B$
  through $\ev: A\otimes B \to I$ and $\coev: I\to B\otimes A$, then the composites
  \begin{gather*}
      FA\otimes FB \xrightarrow{\digamma_2} F(A\otimes B) \xrightarrow{F\ev} FI
      \xrightarrow{\digamma_0} I, 
      \\
      I \xrightarrow{\digamma_0\inv}\to FI \xrightarrow{F\coev} F(B\otimes A)
      \xrightarrow{\digamma_2\inv} FB\otimes FA
  \end{gather*}
  adjoin $FA$ and $FB$ in a duality relation in $\B$.
\end{proof}

\begin{definition}\label{dfn:autonomous-category}
  A monoidal category is left (resp.\ right) autonomous iff every
  object has a left (resp.\ right) dual. It is autonomous iff it is
  both left and right autonomous.
\end{definition}
If $\A$ is an \emph{autonomous} category, denote the dual of an object
$A$ by $\rdl{A}$; the assignment $A \mapsto \rdl{A}$ extends to an
endofunctor of $\A$.

A weaker condition is actually sufficient for braided
monoidal categories to be autonomous.
\begin{proposition}[{\cite[Prop.\ 7.2]{joyal-street;btc}}]
\label{thm:ev-rev}
  Each left autonomous braided monoidal category is autonomous.  
\end{proposition}
\begin{proof}
  By \csref{thm:rev}, we know that the identity induces a monoidal
  isomorphism $\A \to \A\rev$; by \csref{thm:funct-dual} it preserves
  dualtity relations: if $A$ is left dual to $B$ in $\A$ through $\ev:
  A\otimes B \to I$ and $\coev: I\to B\otimes A$, then $A$ is left dual to $B$ in
  $\A\rev$ through
  \begin{equation*}
    \ev' := \ev \circ \tau_{AB} : A\boxtimes B \to I, 
    \qquad
    \coev' := \tau_{AB}\inv \circ \coev : I \to B\boxtimes A,
  \end{equation*}
  from which we see that $A$ is \emph{right} dual to $B$ in $\A$.
\end{proof}

\begin{example}\label{xmp:vect-duality}
  The category $\catVect[\fk]$ of finite-dimensional vector spaces over
  $\fk$ is autonomous with the usual definition of the dual space. If
  $V \in \catVect[\fk]$ and $(e_1, \ldots, e_n)$ is a basis in $V$, $(e^1, \ldots,
  e^n)$ is the dual basis in $\rdl{V}$, then
  \begin{align*}
    \ev &: V \otimes \rdl{V} \ni \sum_i v_i \otimes v^i \mapsto \sum_i \pairing{v_i}{v^i}
    \in \fk \simeq I,
    \\
    \coev &: I \simeq \fk \ni 1 \mapsto \sum_i e^i \otimes e_i \in \rdl{V} \otimes V,
  \end{align*}
  define a (left) autonomous structure on $\catVect[\fk]$. 
\end{example}

\begin{example}\label{xmp:vect-duality-nondeg}
  Let $V \in \catVect$ be a finite dimensional vector space over
  $\fk$. Fix a \emph{non-degenerate} bilinear form $b: V\otimes V \to
  \fk$. Let $\freemc{V}$ be the full subcategory of $\catVect$ which
  has tensor powers $V\tp{p}$ as objects; $\freemc{V}$ inherits a
  symmetric tensor structure from $\catVect$ (see
  \csref{xmp:vec1}). For each $V\tp{p} \in \freemc{V}$ define maps
  \begin{equation*}
    \ev: V\tp{p} \otimes V\tp{p} \ni (v_1, \ldots, v_p) \otimes (v'_1, \ldots, v'_p) \mapsto
    b(v_1, v'_1) \cdots b(v_p, v'_p) \in \fk \simeq I.
  \end{equation*}
  Since $V$ is finite-dimensional, $\ev^\sharp$ is a linear isomorphisms,
  hence a bijection, so $V\tp{p}$ is a self-dual object and
  $\freemc{V}$ is an autonomous category.\FIXME{Esempio gi\`a dato in
    uno dei primi capitoli, togliere da qui?}
\end{example}


\subsection{Balanced and tortile monoidal categories}
\label{sec:tortile}

Let $\A$ be a monoidal category. From \cite{joyal-street;btc} we quote
the following.
\begin{definition}[\cite{joyal-street;btc}]
  A balancing on a monoidal braided category $\A$ is a natural family of
  isomorphisms $\theta_A:A\to A$ such that $\theta_I = \id_I$ and 
  \begin{equation*}
    \xymatrix{%
      A\otimes B
      &
      B\otimes A
      \\
      A\otimes B
      &
      B\otimes A
      \ar "1,1";"1,2" ^{\tau_{AB}}
      \ar "1,1";"2,1" _{\theta_{AB}}
      \ar "2,2";"2,1" _{\tau_{AB}}
      \ar "1,2";"2,2" ^{\theta_B\otimes\theta_A}
      }
  \end{equation*}
  A braided monoidal category equipped with a balancing is called
  balanced.

  A monoidal braided functor $F$ is balanced iff it preserves the
  balancing: $F\theta_A = \theta_{FA}$.
\end{definition}

By \csref{thm:rev}, a braiding $\tau$ determines a monoidal isomorphism
$F: \A \to \A\rev$; the inverse braiding $\tau'_{AB} := (\tau_{BA})\inv$
induces another isomorphism $F': \A \to \A\rev$. A balancing $\theta$ is a
natural equivalence between these two functors.\FIXME{Questo {\`e} l'unico
  paragrafo di motivazioni che ho trovato per giustificare
  l{\'\i}ntroduzione dei bilanciamenti --- nessuno tenta di motivare la
  scoperta di questi o della tortilit{\`a}\ldots}

Now it is immediate to prove the following.
\begin{proposition}
  Any symmetric monoidal category is balanced with the trivial
  balancing $\theta_A := \id_A$.
\end{proposition}

\begin{definition}[\cite{shum;tortile-categories}]
  A balanced category $A$ is called tortile iff it is autonomous and
    $\theta_{A^*} = (\theta_A)^*$ for all $A\in\A$.
\end{definition}


%%% Local Variables: 
%%% mode: latex
%%% TeX-master: "index"
%%% x-symbol-8bits: nil
%%% End: 


