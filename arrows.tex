%auto-ignore


\chapter{Arrows-only description of Categories}
\label{cha:arrows}

It is convenient to have an arrows-only description of a category: in
this picture, an object $A \in \A$ is identified with the identity map
$\id_A \in \A(A, A)$.

According to \cite{lawvere;1965}, a category $\A$ can be described by
the following data:
\begin{itemize}
\item a set $\Hom\A$: the morphisms of $\A$;
\item a ternary relation $\Gamma(x,y;z)$, to be interpreted as ``$z$ is
  the composition of $x$ and $y$'';
\item set-theoretic maps $s,t: \Hom\A \to \Hom\A$ (called \emph{source}
  and \emph{target}), sending an arrow to its domain or codomain
  object,
\end{itemize}
which are tied by the following relations:
\begin{enumerate}
\item $\exists z: \Gamma(x,y;z) \Leftrightarrow s(x) = t(y)$; \label{item:A1}
\item $\Gamma(x,y;z) \land \Gamma(x,y;z') \Rightarrow z = z'$;
\suspend{enumerate}
--- by virtue of this we can adopt the more familiar notation $z = x\circ
y$ for $\Gamma(x,y; z)$, so that the next axioms look more like the
familiar properties of the composition product in a category ---
\resume{enumerate}
\item $z = x\circ y \Rightarrow s(z) = s(y) \land t(z) = t(x)$;
\item $(x\circ y) \circ z = x \circ (y \circ z)$;
\suspend{enumerate}
and, finally, two relations which are useful in characterizing
identity arrows:
\resume{enumerate}
\item $s \circ t = t \circ t = t$;
\item $t \circ s = s \circ s = s$.
\end{enumerate}

We are now ready to introduce objects of $\A$, i.e., identity
arrows. 
\begin{definition}\label{dfn:object}
  $A\in \A$ means that $A$ is a morphism of the category $\A$
  such that:
  \begin{enumerate}
  %\item $\exists x: [A = s(x)] \lor [A = t(x)]$,
  \item $A = s(A) = t(A)$, \label{item:AO1}
  \item $\forall x\forall y[ y = x \circ A \Rightarrow x = y] \land \forall x\forall y [
    x = A \circ y \Rightarrow x = y]$. \label{item:AO2} 
  \end{enumerate}
\end{definition}
Accordingly, we write $A$ for the identity morphism $\id_A$.

It is easy to deduce this characterization from the usual axioms of
category theory, but one can also assume it as a starting point for a
purely categorical foundation of set theory --- see
\cite{lawvere;1965}. 

In this framework, a functor $F: \A \to \category{B}$ is just a map $F:
\Hom\A \to \Hom\B$ such that
\begin{inparaenum}
\item $F(x \circ y) = F(x) \circ F(y)$, and
\item for any object $A \in \A$, $F(\id_A) = \id_{F(A)}$.
\end{inparaenum}

\begin{example}\label{xmp:braid-by-arrows}
  The $\catBraid$ category (\csref{xmp:braids}) is the category whose
  set of morphisms $\Hom\catBraid$ is the set-theoretic union $\bigcup_n
  B_n$ of all braid groups; two morphisms $x$ and $y$ compose iff they
  belong to the same braid group, that is, we can recast \ref{item:A1}
  into
  \begin{equation*}
    \exists z: \Gamma(x,y;z) \Leftrightarrow s(x) = t(y) \Leftrightarrow \exists n: [x, \in B_n \land y \in B_n].
  \end{equation*}
  Then, from \ref{item:AO1} of \csref{dfn:object} we get for an object
  $E \in \catBraid$
  \begin{equation*}
    \exists n: E \in B_n,
  \end{equation*}
  and \ref{item:AO2} says that $E = E_n$ is the unit of some braid group
  $B_n$. Therefore, objects of $\catBraid$ are in $1-1$ correspondence
  with natural numbers.
\end{example}

%%% Local Variables: 
%%% mode: latex
%%% TeX-master: "index"
%%% End: 
