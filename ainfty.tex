%auto-ignore


\chapter{Construction of $A_\infty$ classes}
\label{cha:ainfty}

The construction of cohomology classes on $\M_{g,n}$ from the data of
an $A_\infty$-algebra was first sketched by Kontsevich in the 1994 talk at
ECM \cite{kontsevich;feynman}. It was further explained in
\cite{penkava-schwarz} and \cite{penkava;graph-complexes}, and
generalized in \cite{markl;cyclic} to graph complexes over any cyclic
operad.  A detailed exposition can be found in \cite{conant-vogtmann}.

Here we give a summary of the construction, restricted to the ribbon
graph triangulation of moduli spaces of curves.

This chapter starts with an introduction to the $A_\infty$ structures,
through their operadic formulation. The accent is on generalising the
well-known results for associative algebras; the presentation closely
follows \cite{murri;infty-alg}.  The operadic approach to infinity
algebras was started by Hinich and Schechtman in
\cite{hinich-schechtman;homotopy-lie-algebras}; they further developed
the issues presented in that paper in the later
\cite{hinich;homology-homotopy-algebras} and
\cite{hinich;deformation-homotopy-algebras}. The operadic viewpoint
became very popular and has been adopted by many authors; let us just
mention \cite{ginzburg-kapranov;koszul-duality},
\cite{markl;homotopy-algebras-are-homotopy-algebras},
\cite{markl;homotopy-algebras-via-resolution-of-operads}, which most
directly relate with the contents of this chapter.


\section{$A_\infty$-algebras}
\label{sec:anfty}

\newcommand{\anfty}{\ensuremath{A_\infty}}

\anfty-algebras are a generalization of homotopy associative algebras,
that is, algebras where \(a(bc) - (ab)c = \ell(a,b,c)\) for some
``homotopy'' \(\ell\); for this reason they are also called
``\emph{strongly} homotopy associative algebras.''

Let \(A\) be a \(\k\)-vector space.

\begin{definition}
  A structure of \(A_p\)-algebra on \(A\) is given by a collection of
  \(\k\)-linear maps \(\{m_n \in \Hom_\k^{n-2}(A^{\otimes n},
  A)\}_{1\leq n < p+1}\)  that satisfy the following set of
  relations (\(1\leq n < p+1\)): 
  \begin{equation}
    \label{eq:An+1}
    \sum_{i+j=n+1} \sum_{s=0}^{i-1} \pm m_i(a_1, \dots, a_s,
      m_j(a_{s+1}, \dots, a_{s+j}), a_{s+j+1}, \dots, a_n) = 0,
      \tag{$A_{n+1}$}
  \end{equation}
  the \(\pm\) sign being:
  \begin{equation*}
    \pm = (-1)^{j + s + js + j(a_1+\dots+a_s)}.
  \end{equation*}
\end{definition}

Let us spell out the first three \eqref{eq:An+1} relations:
\begin{align}
  \label{eq:A2}
  0 &= m_1^2 \tag{$A_{2}$} \\
  \label{eq:A3}
  0 &= m_1(m_2(a_1,a_2)) - m_2(m_1(a_1), a_2) - (-1)^{a_1} m_2(a_1,
  m_1(a_2)) \tag{$A_{3}$} \\
  \label{eq:A4}
  \tag{$A_{4}$} 
  0 &=   
  {\begin{aligned}[t]
    -m_1(m_3(a_1,a_2&, a_3)) + m_2(m_2(a_1, a_2), a_3) - m_2(a_1,
    m_2(a_2, a_3)) \\ &- m_3(m_1(a_1), a_2, a_3) - (-1)^{a_1}
    m_3(a_1, m_1(a_2), a_3)  \\ &\phantom{m_3(m_1(a_1), a_2, a_3) +}
    - (-1)^{a_1+a_2} m_3(a_1, a_2, m_1(a_3)) 
  \end{aligned}}
\end{align}
The first one tells us that \(\del := m_1\) is a differential on
\(A\), compatible with the multiplication \(m_2:A^{\otimes 2}\to A\)
by \eqref{eq:A3}. If we rewrite \eqref{eq:A4} as
\begin{multline}
  \label{eq:A4prime}
  m_2(m_2(a_1,a_2),a_3) - m_2(a_1,m_2(a_2,a_3)) = \\ \del
  m_3(a_1,a_2,a_3) + m_3(\del a_1, a_2, a_3) + (-1)^{a_1} m_3(a_1,
  \del a_2, a_3) + \\ + (-1)^{a_1+a_2}m_3(a_1, a_2, \del a_3),
  \tag{$A_4'$}
\end{multline}
then we see that \(m_2\) is associative up to an homotopy given by
\(m_3\); we get a truly associative algebra if \(m_3\equiv 0\), but
the converse needs not be true: i.e., the right-hand side of
\eqref{eq:A4prime} can be 0 without \(m_3\) being trivial---see for
instance the example in section \ref{sec:anfty-nontrivial}.

Summing up, an \(A_1\) algebra is just a \(\k\)-dg-module, an
\(A_2\)-algebra is a \(\k\)-dg-algebra (not necessarily associative),
an \(A_3\)-algebra is a homotopy associative \(\k\)-dg-algebra.

\begin{example}
  Any differential graded associative algebra is an \(A_3\)-algebra
  with \(m_3=0\). Any graded associative algebra is an \(A_3\)-algebra
  with \(m_1=m_3=0\).
\end{example}

\begin{example}[{\cite[p. 5]{khalkhali;homology-infty-algebras}}]
  Given an \(A_\infty\)-algebra \((A, m_*)\) and an associative
  algebra \(B\) (concentrated in degree 0) define a new
  \(A_\infty\)-algebra \((A\otimes B, m')\) by \((A\otimes B)_p := A_p
  \otimes B\) and
  \begin{equation*}
    m'_n (a_1\otimes b_1, \dots, a_n\otimes b_n) := m_n(a_1, \dots,
    a_n) \otimes (b_1\cdots b_n).
  \end{equation*}
\end{example}

\begin{example}
  Given an associative \(\k\)-algebra \(A\), graded over \(\setZ/2\setZ\)
  and concentrated in degree 0, pick \(\mu_2, \mu_4, \dots, \mu_{2t}
  \in \k\) and define:
  \begin{align*}
    m_{2i+1}(a_1, \dots, a_{2q+1}) &:= 0 &q=0,1,\dots; \\
    m_{2r}(a_1, \dots, a_{2r}) &:= \mu_{2r} \cdot a_1\cdots
    a_{2r} &1\leq r \leq t; \\
    m_{2s}(a_1,\dots, a_{2s}) &:= 0 & s>t;
  \end{align*}
  Then \((A, m'_*)\) is an \(A_\infty\)-algebra.
\end{example}

Note that we may rewrite \eqref{eq:An+1} as
\begin{multline}
  \label{eq:An+1prime}
  \sum_{\substack{i+j=n+1 \\ i,j \geq 2}} \sum_{s=0}^{i-1} \pm m_i(a_1, \dots,
  a_s, m_j(a_{s+1}, \dots, a_{s+j}), a_{s+j+1}, \dots, a_n)
  \\
  = [m_n, \del](a_1, \dots, a_n), \tag{$A_{n+1}'$}
\end{multline}
where \(\del = m_1\) and \([\cdot,\cdot]\) is the Lie bracket on the
Hochschild complex:
\begin{multline*}
  [m_n,\del](a_1,\dots,a_n) := \sum_{s = 0}^{n-1} (-1)^{a_1+\dots+a_s}
  m_n(a_1, \dots, a_s, \del a_{s+1}, a_{s+2}, \dots, a_n) + \\ 
  +(-1)^{n+1} \del m_n(a_1, \dots, a_n).
\end{multline*}
The right-hand side of \eqref{eq:An+1prime} vanishes on passing to
\(\del\)-cohomology: therefore, \(H_\del(A)\) inherits a structure of
\(A_n\)-algebra with trivial differential.


\subsection{A less trivial example}
\label{sec:Anfty-nontrivial}

In \cite{zhou;hodge-theory-infty-structures}, J.\ Zhou adapted a
construction by S.\ A.\ Merkulov to give an \anfty-structure on the
space of harmonic forms on a Riemannian manifold. Here's a sketch of
their construction.

Let \(V\) be a DGA over \(\k\) with differential \(\d\); let \(W\) be
a vector subspace of \(V\) such that \(\d W \subset V\); let \(Q:V \to
V\) be an \emph{odd} operator such that \(P:=(\Id-[\d,Q])\) has range
lying in \(W\). Define \(\k\)-linear maps \(\lambda_n:V^{\otimes n}
\to V\) by 
\begin{equation*}
  \lambda_2(v_1, v_2) := v_1 \cdot v_2
\end{equation*}
where \(\cdot\) is the ordinary multiplication in \(V\), and then,
recursively, for \(n\geq 3\):
\begin{multline*}
  \lambda_n (v_1, \dots, v_n) := (-1)^{n-1} \bigl(Q\lambda_{n-1} (v_1,
  \dots, v_n) \bigr) \cdot v_n + \\ - \sum_{\substack{k+l=n+1 \\
      k,l \geq 2}} (-1)^{k+(l-1)(v_1+\dots+v_k)} Q\lambda_k(v_1,
  \dots,v_k) \cdot Q\lambda_l(v_{k+1}, \dots, v_n) + \\ - (-1)^{nv_1}
  v_1 \cdot \bigl( Q \lambda_{n-1}(v_2, \cdot, v_n) \bigr)
\end{multline*}

Then a longish, yet direct, computation proves the following
statement.
\begin{theorem}[Merkulov, \cite{merkulov;strong-homotopy-algebras}]
  The \(\k\)-linear maps \(m_n:A^{\otimes n} \to A\) defined by
  \begin{align*}
    m_1 :&= \d, \\
    m_n :&= P\circ \lambda_n, \quad n\geq 2, 
  \end{align*}
define an structure of \anfty-algebra on \(W\).
\end{theorem}

As an immediate corollary, we get an \anfty-algebra structure on the
space of harmonic forms on a Riemannian manifold \(X\): take \(V =
\mathcal{E}^*(X)\), the DGA of differential forms with multiplication given by
the wedge product, \(W=\Harmonic^*(X)\) the space of harmonic forms,
and \(Q=G\d^*\) where \(\d = -*\circ \d \circ *\) and \(G\) is the Green
operator. Then \(P = \Id - G\d^*\d -\d G\d^*\) is the projector on the
space \(\Harmonic^*(X)\), and the multiplication \(m_2: \Harmonic^*(X)
\to \Harmonic^*(X)\) takes two forms to the harmonic part of their
wedge product. It can be shown (see
\cite{zhou;hodge-theory-infty-structures}) that \(m_2\) is
associative; therefore, by \eqref{eq:A4prime}, \([d,m_3] = 0\), yet this
does not imply that \(m_3 = 0\).

Passing to homology with respect to \(\d\), we get an \anfty-structure
on \(H_\d(\Harmonic^*(X)) = H_{\textrm{dR}}^*(X)\) with trivial
differential; so the left-hand side of \eqref{eq:An+1prime} vanishes, in
particular, \(m_2\) is an associative multiplication---indeed it is
the usual cup product.

Similarly, one can define \anfty-algebra structures on the Dolbeault
cohomology of any complex manifold \(X\); for this and other examples,
see \cite{zhou;hodge-theory-infty-structures,
  merkulov;strong-homotopy-algebras}. 


\section{The $\Ao_\infty$ operad}
\label{sec:anfty-operad}
We can build an operad $\Ao_\infty$ which governs $A_\infty$-algebras,
paralleling the construction of the $\Ao$ operad of associative
algebras. 

Recall from \csref{sec:anfty} that an $A_\infty$-algebra has
multiplications $m_1, m_2, m_3, \ldots$ related by \eqref{eq:An+1}.
Let us use the same graphical notation of
\csref{sec:operad-assoc}; we still depict the multiplication
$m_2(x_1, x_2)$
by the two-branched tree
\begin{equation*}
  \begin{xytree}
    \branch{%
      \leaf{1}
      \leaf{2}
      }
  \end{xytree}
\end{equation*}
However,  higher-order operations are no longer compositions of binary
multiplications: we have primitive $n$-ary maps, for any $n\geq1$;
therefore, it natural to associate the homotopy $m_n$ with the
$n$-corolla
\begin{equation*}
  \begin{xytree}
    \branch{%
      \leaf{1}
      \leaf{2}
      \gap[\ldots]
      \leaf{n}
      }
  \end{xytree}
\end{equation*}
Define the composition $\gamma(t \otimes t_1 \otimes \ldots \otimes t_n)$ of trees $t$, with
$n$ leaves, and $t_j$, with $k_j$ leaves each, to be the tree in which
$t_j$ has been grafted onto the $j$-th leaf of $t$:
\begin{equation*}
  \gamma(t \otimes t_1 \otimes \ldots \otimes t_n) = \xytreec
  \bbranch{t}{%
    \leaf{t_1}
    \leaf{t_2}
    \gap[\ldots]
    \leaf{t_n}
    }
  \endxytreec
\end{equation*}
By repeated composition of corollas, we can form \emph{any} tree.
These trees have numbered leaves; define the action of $\Perm{n}$ on
the set of $n$-leaved trees by permutation of numbers on leaves.

We know from \eqref{eq:A2} that $m_1$ is a differential on any
$A_\infty$-algebra $X$; therefore, a
standard calculation\FIXME{Federico: qui ci va un riferimento alla tua
  parte su Hochschild} shows that $\ad(m_1) := [m_1, -]$ is a
differential in the Hochschild complex of $X$. Since $\Ao_\infty$ ought to
be an operad in the category of $R$-dg-modules, we define a
differential $D$ on $\Ao_\infty(n)$ by $D := \ad(m_1)$; the fundamental
relation \eqref{eq:An+1prime} translates into:
\begin{equation}
  \label{eq:anfty-D-1}
  D\left( \xytreec\branch{\leaf{1}\gap[\ldots]\leaf{n}}\endxytreec \right) =
  \sum_{\substack{i+j=n+1 \\ i,j \geq 2}} \sum_{s=0}^{i-1} \pm
  \xytreec\branch{%
    \leaf{a_1}
    \gap[\ldots]
    \leaf{a_s}
    \branch{%
      \leaf{a_{s+1}}
      \gap[\ldots]
      \leaf{a_{s+j+1}}
      }
    \gap[\ldots]
    \leaf{a_n}
    }\endxytreec, 
  \qquad n\geq3.
\end{equation}
The Leibniz rule \eqref{eq:A3} now becomes:
\begin{equation}
  \label{eq:anfty-D-2}
  D\left( \xytreec\branch{\leaf{1}\leaf{2}}\endxytreec \right) = 0.
\end{equation}

Summing up, we can give the following definition.
\begin{definition}
  The operad $\Ao_\infty$ is the operad defined by the collection
  $\{\Ao_\infty(n)\}$, where $\Ao_\infty(n)$ is the $R$-dg-module freely
  generated by set of all trees with $n$ leaves, numbered from $1$ to
  $n$, equipped with the differential defined on generators by
  \eqref{eq:anfty-D-1} and \eqref{eq:anfty-D-2}.  The structure maps
  are $R$-multilinear extensions of the grafting operation $\gamma$ (see
  \eqref{eq:grafting}). Finally, $\Perm{n}$ acts on $\Ao_\infty(n)$ by
  permuting the numbers on leaves.
\end{definition}
It is important to note that, in contrast with the associative operad
$\Ao$, there no longer is any relation among the trees in $\Ao_\infty(n)$,
that is, $\Ao_\infty$ is a \emph{free} operad (generated by the set of
corollas) and all relations are encoded into the differential. (This
has been singled out by Markl as a characteristics of ``homotopy
algebras'' operads, see
\cite{markl;homotopy-algebras-via-resolution-of-operads,
  markl;models}.)

The relation between the associative and the $A_\infty$ operad is
summarized in the following.
\begin{theorem}
  \label{prop:surjection-ass}
  There is a surjective morphism $\Ao_\infty \to \Ao$ in the category of
  differential graded $R$-operads.
\end{theorem}
\begin{proof}
  Define a morphism $\phi: \Ao_\infty \to \Ao$ on the generators by
  \begin{align*}
    &\phi\left( \xytreec\branch{\leaf{1}\leaf{2}}\endxytreec \right) =
    \xytreec\branch{\leaf{1}\leaf{2}}\endxytreec, 
    \\
    &\phi\left( \xytreec\branch{\leaf{1}\gap[\ldots]\leaf{n}}\endxytreec
    \right) = 0, \qquad n>2,
  \end{align*}
  and extend it to be $R$-linear and a morphism of
  operads. 

  Compatibility with the differential $D$ requires 
  \begin{equation*}
    \phi \circ D = D \circ \phi,
  \end{equation*}
  which again needs only be checked on generators. Now, $D$ of a
  $n$-corolla is given by \eqref{eq:anfty-D-1}; observe that, for
  $n>3$ on RHS we always see the $2$-corolla paired with some other
  $>2$-corolla --- since $\phi$ is an operad morphism, this dooms it to
  be zero. 

  So we are left with verifying compatibility only for $2$-
  and $3$- corollas. Apply $\phi$ to both sides of 
  \begin{equation*}
    D_{\Ao_\infty}\left( \xytreec\branch{\leaf{1}\leaf{2}\leaf{3}}\endxytreec
    \right) =
    \xytreec\branch{\leaf{1}\branch{\leaf{2}\leaf{3}}}\endxytreec 
    \pm
    \xytreec\branch{\branch{\leaf{1}\leaf{2}}\leaf{3}}\endxytreec 
  \end{equation*}
  to get
  \begin{equation*}
    \phi \circ D_{\Ao_\infty}\left( \xytreec\branch{\leaf{1}\leaf{2}\leaf{3}}\endxytreec
    \right) =
    \xytreec\branch{\leaf{1}\branch{\leaf{2}\leaf{3}}}\endxytreec 
    \pm
    \xytreec\branch{\branch{\leaf{1}\leaf{2}}\leaf{3}}\endxytreec 
    = 0,
  \end{equation*}
  because the $2$-corolla satisfies the associativity relation
  \eqref{eq:ass-tree}. On the other hand,
  \begin{equation*}
    D_\Ao \circ \phi \left( \xytreec\branch{\leaf{1}\leaf{2}\leaf{3}}\endxytreec
    \right) = D_\Ao (0) = 0.
  \end{equation*}
  The remaining verification is trivial.
\end{proof}

As an immediate corollary to the above and the structure transfer
theorem \ref{prop:structure-transfer}, we get the following.
\begin{corollary}
  Every associative algebra is an $A_\infty$-algebra.
\end{corollary}



%%% Local Variables: 
%%% mode: latex
%%% TeX-master: "index"
%%% End: 
