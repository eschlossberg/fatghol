\RCSID $Id: construction.tex,v 1.2 2006/05/18 08:07:28 rmurri Exp $
%auto-ignore


\chapter{Construction of $A_\infty$ classes}
\label{cha:construction}

The construction of cohomology classes on $\M_{g,n}$ from the data of
an $A_\infty$-algebra was first sketched by Kontsevich in the 1994 talk at
ECM \cite{kontsevich;feynman}. It was further explained in
\cite{penkava-schwarz} and \cite{penkava;graph-complexes}, and
generalized in \cite{markl;cyclic} to graph complexes over any cyclic
operad.  A detailed exposition can be found in \cite{conant-vogtmann}.

Here we give a summary of the construction, restricted to the ribbon
graph triangulation of moduli spaces of curves.


\section{Arc-systems and their complexes}
\label{sec:arc-systems}

Harer \cite{harer;cohomological-dimension}, introduced the arc-systems
complex as a tool to calculate the (co)homology of the moduli space
$\M_{g,n}$.

Let $S_{g,n}$ be a topological compact oriented Riemann surface of
fixed genus $g$ with $n$ marked points $x_1, \ldots, x_n$.  Points in the
moduli space $\M_{g,n}$ can be regarded as equivalence classes
$(C,x,f)$ of a complex curve $C$ with a marking $x:\{1,\ldots,n\}\to C$ and a
homeomorphism $f:C\to S_{g,n}$ with the reference surface (see
\csref{rem:moduli-with-reference}).

If $\alpha$ is any arc in $C$, denote by $[\alpha]$ its isotopy class rel~$\{x_1,
\ldots, x_n\}$.

\begin{definition}
  An arc-system $[\alpha_1, \ldots, \alpha_k]$ of rank $k$ is an isotopy class
  rel~$\{x_1, \ldots, x_n\}$ of $k$ properly imbedded arcs $\alpha_i : [0,1] \to C$
  such that:
  \begin{itemize}
  \item every $\alpha_i$ has its endpoints in the set $\{x_1, \ldots, x_n\}$;
  \item if $i \neq j$, then $\alpha_i$ meet $\alpha_j$ only at endpoints, if at all;
  \item no $\alpha_i$ is null-homotopic rel~$\{x_1, \ldots, x_n\}$;
  \item no $\alpha_i$ is homotopic rel~$\{x_1, \ldots, x_n\}$ to $\alpha_j$, for $i \neq
    j$;
  \end{itemize}
\end{definition}

An arc-system $[\alpha_1, \ldots, \alpha_k]$ is said to \emph{fill up} the curve $C$
iff all connected components of $C - \bigcup\{\alpha_i\} - \{x_1,\ldots x_n\}$ are either
disks or punctured disks.

Arc-systems may be organized into a semi-simplicial complex.
\begin{definition}
  $X$ is the semi-simplicial complex having a simplex $\langle\alpha_1, \ldots, \alpha_k\rangle$
  for every rank $k$ arc-system $[\alpha_1, \ldots \alpha_k]$, with the proviso that
  $\langle\beta_1, \ldots, \beta_l\rangle$ is a face of $\langle\alpha_1, \ldots, \alpha_k\rangle$ iff $\{ [\beta_1], \ldots, [\beta_l]
  \} \subset \{ [\alpha_1], \ldots, [\alpha_k] \}$.
  
  $X_\infty$ is the subcomplex of all those simplices $\langle\alpha_1, \ldots, \alpha_k\rangle$
  that do not fill up the surface $S$.
\end{definition}

Let $\Delta^{n-1}$ be the $(n-1)$-dimensional geometric simplex.
\begin{theorem}[Harer, {\cite[Theorem
  1.3]{harer;cohomological-dimension}}]
There exists a $\Gamma_{g,n}$-equivariant homeomorphism $\Phi: \T_{g,n} \times
\Delta^{n-1} \to X \setminus X_\infty$.
\end{theorem}

Let $X^\circ$, $X_\infty^\circ$ be the first barycentric subdivisions of $X$
and $X_\infty$, respectively.

\begin{definition}
  Let $Y^\circ$ be the collection of simplices in $X^\circ$ having no face in
  $X_\infty^\circ$; equivalently, $\sigma_\bullet \in Y^\circ$ iff every vertex of $\sigma_\bullet$
  lies in $X^\circ \setminus X_\infty^\circ$.  $Y^\circ$ is a full subcomplex of $X^\circ$.
\end{definition}


%%% Local Variables: 
%%% mode: latex
%%% TeX-master: "index"
%%% End: 
