%% <<filename>>
%%
%% Report on the fatgraphs making up the Kontsevich graph complex of
%% the moduli space M_${<<g>>,<<n>>}$ of punctured Riemann surfaces.
%%
%% Generated by FatGHoL's "latex" command (version <<version>>)
%% from data in '<<checkpoint_dir>>'.
%%
\documentclass[a4paper]{article}
\raggedbottom

% use Knuth's "concrete" fonts, which blend better with the
% typewriter font used for code listings
\usepackage[T1]{fontenc}
\usepackage[boldsans]{ccfonts}

\usepackage{ifpdf}
\ifpdf
  % maximum compression for PDF output
  \pdfcompresslevel9
\fi

\usepackage{amsmath}
\usepackage{amsfonts}
\usepackage{colortbl}
\usepackage{hyperref}
\usepackage{longtable}%
  \setcounter{LTchunksize}{100}%
\usepackage{multicol}
\usepackage{tensor}
\usepackage[usenames,dvipsnames]{xcolor}

\ifpdf 
  % NOTE: Xy-PIC 3.8 is needed for PDF support!
  \usepackage[pdf,color,curve,frame,line,poly]{xy}
\else  
  % fall-back to (x)DVI \specials
  \usepackage[xdvi,color,curve,frame,line,poly]{xy}% 
  \UseCrayolaColors
\fi  

  
\newcommand{\corner}[3]{\ensuremath{\tensor[^{#2}]{#1}{^{#3}}}}
\newcommand{\cornerjoin}{\to}
% alternate:
%\newcommand{\corner}[3]{\ensuremath{\stackrel{#2}{-}#1\stackrel{#3}{-}}}
%\newcommand{\cornerjoin}{\kern-0.25ex\to\kern-0.25ex}


%% disable section numbering
\setcounter{secnumdepth}{-1}


\begin{document}

\title{Fatgraphs of $M_{<<g>>,<<n>>}$}
\author{%%
  Automatically generated by FatGHoL <<version>>
  \\
  (See: \url{http://fatghol.googlecode.com/})
  }
\date{ <<today>> }
\hypersetup{%%
  pdftitle={Fatgraphs of M<<g>>,<<n>>},%%
  pdfcreator={FatGHoL},%%
  pdfkeywords={{fatgraphs} {M<<g>>,<<n>>}
               {moduli space} {associative operad}},%%
  }
\maketitle


\noindent{}
There are a total of <<total_num_graphs>> undecorated fatgraphs in the
Kontsevich graph complex of $M_{<<g>>,<<n>>}$, originating
<<total_num_marked_graphs>> marked ones.


% only insert 'Fatgraphs with XXX edges' entries
\setcounter{tocdepth}{1}
\tableofcontents

\clearpage


\section{Notation}

We denote $G_{m,j}$ the $j$-th graph in the set of undecorated
fatgraphs with $m$ edges; the symbol $G_{m,j}^{(k)}$ denotes the
$k$-th inequivalent marking of $G_{m,j}$.

Fatgraph vertices are marked with lowercase latin letters
``a'', ``b'', ``c'', etc.; edges are marked with an arabic
numeral starting from ``1''; boundary cycles are denoted
by lowercase greek letters ``$\alpha$'', ``$\beta$'', etc.

Automorphisms are specified by their action on the set of vertices,
edges, and boundary cycles: for each automorphism $A_k$, a table line
lists how it permutes vertices, edges and boundary cycles relative to
the identity morphism $A_0$.  The automorphism table is printed only
if the automorphism group is non-trivial.

Automorphisms that reverse the orientation of the unmarked fatgraph
are indicated with a ``$\dag$'' symbol in the automorphism table;
those that reverse the orientation of the marked fatgraphs are
distinguished with a ``$\ddag$'' sign.

If a fatgraph is orientable, a ``Markings'' section lists all the
inequivalent ways of assigning distinct numbers $\{0, \ldots, n-1\}$
to the boundary cycles; this is of course a set of representatives for
the orbits of $\mathfrak{S}_n$ under the action of $\mathrm{Aut}(G)$.

A separate section lists the differential of marked fatgraphs; graphs
with null differential are omitted.  If no marked fatgraph has a
non-zero differential, the entire section is dropped.

Boundary cycles are specified using a ``sequence of corners''
notation: each corner is represented as \corner{L}{p}{q} where $L$ is
a latin letter indicating a vertex, and $p$,~$q$ are the attachment
indices of the incoming and outgoing edges, respectively.  Attachment
indices match the Python representation of the vertex: e.g., if
a$=$\verb'Vertex([0,0,1])', the two legs of edge~$0$ have attachment
indices~0 and~1, and the boundary cycle enclosed by them is
represented by the (single) corner~\corner{a}{0}{1}.

\clearpage


<<for section in sections>>
\section{<<section.title>>}
<<section.intro>>

<<if section.num_graphs > 1>>%
There are <<section.num_graphs>> unmarked fatgraphs in this section,
<<else>>% only one graph
There is 1 unmarked fatgraph in this section,
<<endif>>%
<<if section.num_marked_graphs > 1>>
originating <<section.num_marked_graphs>> marked fatgraphs 
(<<if section.num_orientable_marked_graphs > 0>>%
  <<if section.num_orientable_marked_graphs == section.num_marked_graphs>>%
    all of them
  <<else>>%
    <<section.num_orientable_marked_graphs>>
  <<endif>> %
  orientable%
 <<endif>>%
 <<if section.num_orientable_marked_graphs * section.num_nonorientable_marked_graphs != 0>>%
 , and %
 <<endif>>%
 <<if section.num_nonorientable_marked_graphs > 0>>%
   <<if section.num_nonorientable_marked_graphs == section.num_marked_graphs>>%
   all of them
   <<else>>%
     <<section.num_nonorientable_marked_graphs>>
   <<endif>> %
   nonorientable%
 <<endif>>).
<<else>>% 1 marked counterpart as well
  <<if section.num_orientable_marked_graphs > 0>>%
    originating 1 orientable marked fatgraph.
  <<else>>
    originating 1 non-orientable marked fatgraph.
  <<endif>>
<<endif>>

<<for G in section.graphs >>
\vspace{2em}
\subsection{The Fatgraph $<<G.name>>$ %
  {\em (<<if not G.orientable>>non-orientable, <<endif>>%
    <<if G.num_orientable_markings == 0>>%
      no
    <<else>>%
      $<<G.num_orientable_markings>>$
    <<endif>> %
    orientable marking<<if G.num_orientable_markings != 1>>s<<endif>>)}}%
\vspace{-1em}
\begin{tabular}{lr}
  \begin{minipage}{0.45\textwidth}
  {% left column: Xy-Pic diagram
\hspace{-3em}
<<G.latex_repr>>
  }%
  \end{minipage}
  &% right column: Python code
  \begin{minipage}{0.45\textwidth}
\begin{verbatim}
<<G.python_repr>>
\end{verbatim}
  \end{minipage}
\end{tabular}
\vspace{-2em}

\subsubsection{Boundary cycles}
<<G.boundary_cycles>>

<<if 'automorphisms' in G>>
\subsubsection{Automorphisms}
<<G.automorphisms>>
<<endif>>

<<if 'markings' in G>>
\subsubsection{Markings}
<<G.markings>>
<<endif>>

<<if 'differentials' in G>>
\subsubsection{Differentials}
<<G.differentials>>
<<endif>>

<<endfor>>%% end graph


<<endfor>>%% end sections


<<if appendices>>
\appendix

<<if appendix_graph_markings>>
\section{Markings of fatgraphs with trivial automorphisms}
\label{appendix:markings}

This appendix shows the numbering of marked fatgraphs when the base
unmarked fatgraph $G$ has only the trivial automorphism.

<<appendix_graph_markings>>
<<endif>>%% appendix graph markings

<<endif>>%% end appendices


\end{document}


%%% Local Variables: 
%%% mode: latex
%%% TeX-master: t
%%% End: 
