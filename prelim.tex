\RCSID $Id: prelim.tex,v 1.6 2006/05/18 08:07:29 rmurri Exp $
%auto-ignore


\chapter{Preliminaries and Notation}

When $\A$ is a category, we shall use Eilenberg's $\A(A, B)$ to denote
the set of arrows from $A$ to $B$, in place of the more verbose $\Hom
\A(A, B)$.  The words ``morphism'', ``map'' and ``arrow'' will be used
interchangeably.  By abuse of notation, we write $A \in \A$ to mean that
``$A$ is an object of $\A$''. Often, the notation $\Hom\A$ will be
used for the union of all the $\Hom$-sets of the category $\A$, i.e.,
$\Hom\A := \bigcup_{A,B \in \A} \A(A,B)$.

The symbol $\Perm{k}$ stands for the group of permutations on $k$
letters.

The degree $i$ component of a graded object $M^\#$ will be denoted
$M^i$; the grading group will always be $\setZ$. Elements of a graded
object appearing as exponents to a number will stand for their degree,
i.e., $(-1)^a := (-1)^{\deg a}$; hence, $(-1)^{ab} \not=
(-1)^{\deg (ab)} = (-1)^{\deg a + \deg b} = (-1)^{a+b}$.


\section{Differential graded algebras and modules}
\label{sec:dg-things}

Let $\fk$ be a fixed field of characteristics $0$; later on, we shall
assume $\fk = \setC$.
\begin{definition}
  A dg-algebra $K$ consists of a graded vector space $K^\#$ over $\fk$,
  a $\fk$-bilinear associative product, and a $\fk$-linear map $\ud:
  K\to K$ such that:
  \begin{enumerate}
  \item $K^i \cdot K^j \subset K^{i+j}$;
  \item $ab = (-1)^{ab} ba$ (graded commutativity);
  \item $\ud(K^i) \subset K^{i+1}$;
  \item $\ud(a\cdot b) = \ud(a)\cdot b + (-1)^a a\cdot\ud(b)$ (graded Leibniz rule).
  \end{enumerate}
  Furthermore, we shall always assume that a dg-algebra has a unit
  $1\in K^0$ such that $1\cdot a = a\cdot 1 = a$ for all $a\in
  K$.
\end{definition}

Any dg-algebra is a (graded) associative commutative algebra.

\begin{definition}
  A $K$-dg-module $M$ is a $K$-bimodule such that:
  \begin{enumerate}
  \item $K^i M^j \subset M^{i+j}$, $M^i K^j \subset M^{i+j}$;
  \item $\lambda x = (-1)^{\lambda x} x\lambda$, for all homogeneous
    $x\in M$, and $\lambda\in K$;
  \item $\ud(\lambda x) = \ud(\lambda)x + (-1)^\lambda \lambda\ud(x)$.
  \end{enumerate}
\end{definition}

There is an obvious underlying functor $\#$ that ``forgets the
differential'', from the category of $K$-dg-modules to the category of
$K$-modules.

\begin{definition}
  For every two $K$-dg-modules $M$, $N$ we define the module of
  degree $p$ morphisms
  \begin{equation*}
    \Hom_K^p (M,N) := \Bigl\{ f\in  {\textstyle \prod_i} Hom_\fk (M^i, N^{i+p}) :
      f(x\lambda) = f(x)\lambda \quad \forall x\in M, \forall \lambda
      \in R \Bigr\},
  \end{equation*}
  and the graded module of morphisms
  \begin{equation*}
    \Hom_K^* := \oplus_{p\in\setZ} \Hom_K^p (M,N)[-p],
  \end{equation*}
  that is, $\Hom_K^p(M,N)$ is the degree $p$ component of
  $\Hom_K^*(M,N)$.
\end{definition}

$\Hom_K^*(M,N)$ is made into a $K$-dg-module by the differential:
\begin{equation*}
  \ud f := \ud_N \circ f + (-1)^{p+1} f\circ \ud_M \qquad f\in \Hom_K^p(M,N)
\end{equation*}
The term ``morphism'' will denote any element of $\Hom_K^*$, whereas
``dg-morphism'' will be applied only to those such that $\ud f = 0$.


\section{Braided symmetric categories of Modules}
\label{sec:btc+dg}

Given any commutative dg-algebra $K$, the category $\catMod_K^\setZ$ of
finitely-generated $K$-dg-bimodules is in a natural way braided and
balanced with the structure maps:
\begin{align}
  \label{eq:dgm-braiding}
  \tau_{MN} &: M \otimes N \ni m\otimes n \mapsto (-1)^{mn}n\otimes m \in N \otimes M
  &&
  \text{(braiding),}
  \\
  \label{eq:dgm-balancing}
  \theta_M &: M \ni m \mapsto (-1)^mm \in M
  &&
  \text{(balancing).}
\end{align}
The verification that these maps satisfy the axioms of a balanced
braided symmetric category is an easy calculation.  The usual
definition of the dual object $\rdl{M} := \Hom_K(M, K)$ gives an
obvious evaluation map $\ev_M : M \otimes \rdl{M} \ni m \otimes \rdl{m} \mapsto
\pairing{m}{\rdl{m}} \in K$, but one cannot generally assert the
existence of a matching coevaluation morphism unless $M$ is a free
module.

The above structure maps
\eqref{eq:dgm-braiding}--\eqref{eq:dgm-balancing} induce a balanced
symmetric braided category structure on the subcategory $\catMod_K$ of
degree $0$ $K$-bimodules; the braiding on $\catMod_K$ restricts to the
trivial one: $\tau_{MN}: m\otimes n \mapsto n\otimes m$.
\begin{remark}
  In fact, Eilenberg and Kelly \cite{eilenberg-kelly;closed-categories}
  show that there are just two possible braidings on $\catMod_K^\setZ$:
  the trivial one given by:
  \begin{equation*}
    \tau_{MN}^+ : m\otimes n \mapsto n\otimes m,
  \end{equation*}
  which is the usual braiding on categories of non-graded modules and
  vector spaces, and and that of \eqref{eq:dgm-braiding},
  \begin{equation*}
    \tau_{MN}^- : m\otimes n \mapsto (-1)^{mn} n\otimes m,
  \end{equation*}
  which is the usual one on categories of graded modules.
\end{remark}


\section{Signs and the graded contraction}
\label{sec:signs}

Recall that the symmetric monoidal category of graded modules has a
non-trivial (yet involutive) braiding isomorphism
\begin{equation*}
\tau_{(12)} = \tau_{M_1M_2} : M_1
\otimes M_2 \ni x_1\otimes x_2 \mapsto (-1)^{x_1x_2} x_2\otimes x_1
\in M_2\otimes M_1,
\end{equation*}
for all objects $M_1$, $M_2$.  Any $\sigma\in\Perm{n}$ has a decomposition
into a product of traspositions $(ij)$; since the category of graded
modules is sysmmetric, we can define an isomorphism $\tau_\sigma: M_1 \otimes \cdots \otimes
M_n \to M_{\sigma_1} \otimes \cdots \otimes M_{\sigma_n}$ as the composition of twists $\tau_{(ij)}$;
$\tau_\sigma$ does not depend on the chosen factorization of $\sigma$ into
traspositions.

Now, suppose $M_1 = \dots = M_n = M$, a fixed object: $x_1\otimes \dots \otimes
x_n$ and $\tau_\sigma(x_1\otimes \dots \otimes x_n)$ differ only by the sign; therefore we
define the Koszul sign $\epsilon(\sigma; x_1, \dots, x_n)$ so that the following
holds:
\begin{equation}
  \label{eq:ksz}
  \epsilon(\sigma; x_1, \dots, x_n) x_1 \otimes \dots \otimes x_n =
  \tau_\sigma (x_1 \otimes \dots \otimes x_n), \qquad \sigma\in\Perm{n}.
\end{equation}
Also define the alternating Koszul sign $\chi(\sigma; x_1, \ldots, x_n)$ by:
\begin{equation}
  \label{eq:aksz}
  \chi(\sigma; x_1, \dots, x_n) x_1 \otimes \dots \otimes x_n =
  (-1)^\sigma \varepsilon(\sigma; x_1, \ldots, x_n)
  \qquad \sigma\in\Perm{n}.
\end{equation}
We shall omit $x_1, \dots, x_n$ from the above when it will be clear
from the context which elements $\epsilon$ or $\chi$ are being
applied to.

\begin{definition}\label{dfn:graded-sym-map}
  A map $f: M^{\otimes n} \to N$ is graded symmetric iff
  \begin{equation*}
    f(x_1, \dots, x_n) = \epsilon(\sigma) f(x_{\sigma_1}, \dots,
    x_{\sigma_n}) \qquad \forall x_1, \dots, x_n \in M.
  \end{equation*}
  It is graded antisymmetric iff
  \begin{equation*}
    f(x_1, \dots, x_n) = \chi(\sigma) f(x_{\sigma_1}, \dots,
    x_{\sigma_n}) \qquad \forall x_1, \dots, x_n \in M.
  \end{equation*}
\end{definition}

Since permutation of factors in a tensor product may change the sign,
the contraction map 
\begin{equation*}
c: {\textstyle \bigotimes_i} \Hom(M_i, N_i) \otimes
 {\textstyle \bigotimes_i} M_i \to  {\textstyle \bigotimes_i} N_i
\end{equation*}
is well-defined only up to a sign. In search of a remedy, we fix an
ordering of the factors which gives ``unsigned'' contraction, and then
define contraction of any permutation of the factors
$\left\{\Hom(M_i,N_i), M_i\right\}_{i\in I}$ by first applying a twist
$\tau_\sigma$ to get the fixed ``standard'' ordering, and then the usual
``unsigned'' contraction. Formally:
\begin{enumerate}
\item let $c: \Hom(M,N) \otimes M \ni f\otimes x \mapsto f(x) \in N$;
\item let $c: \bigotimes_i (\Hom(M_i, N_i) \otimes M_i) \ni (f_1\otimes x_1)
  \otimes \dots (f_k\otimes x_k) \mapsto f_1(x_1) \otimes \dots
  \otimes f_k(x_k) \in \bigotimes N_i$;
\item let $W_{2i-1} := \Hom(M_i, N_i)$, $W_{2i} := M_i$, for $i =
  1, \dots, k$; define the contraction
  \begin{equation*}
    c: {\textstyle \bigotimes_{j=1}^{2k}} W_{\sigma_j} \to
    {\textstyle \bigotimes_{i=1}^k} N_i
  \end{equation*}
  by 
  \begin{equation*}
    {\textstyle \bigotimes_{j=1}^{2k}} W_{\sigma_j} \to {\textstyle \bigotimes_{j=1}^{2k}} W_j \to
    {\textstyle \bigotimes_{i=1}^{k}} N_i.
  \end{equation*}
\end{enumerate}

Infact, this boils down to the rule: ``change the sign by
$(-1)^{pq}$ when interchanging objects of degree $p$ and $q$''.


\section{Braided symmetric categories of vector spaces}
\label{sec:btc-vect}

The categories of ($\setZ$-graded) $\fk$-linear spaces, $\catVect[\fk]^\setZ$
and $\catVect[\fk]$ have natural structures of symmetric braided
autonomous tortile category if we consider $\fk$ to be a
$\fk$-dg-algebra concentrated in degree $0$.

However, every vector space is a free module, so, for any $V \in
\catVect^\setZ$, pick a base $(v_i)$ and a dual base $(\rdl{v_i})$ of
$\rdl{V}$ such that $\pairing{v_i}{\rdl{v_j}} = \delta_{ij}$; we can
define evaluation and coevaluation maps
\begin{align}
  \label{eq:vec-evaluation}
  \ev_V &: v \otimes \rdl{v} \mapsto \pairing{v}{\rdl{v}}, 
  \\
  \label{eq:vec-coevaluation}
  \coev_V &: 1 \mapsto \sum_i \rdl{v_i} \otimes v_i,
\end{align}
which make $\catVect^\setZ$ into an autonomous category; these morphisms
are actually independent of the chosen basis $(v_i)$. The balancing
map \eqref{eq:dgm-balancing} is compatible with taking duals, so
$\catVect^\setZ$ is a tortile category.
\begin{remark}\label{rem:ev-rev}
  \FIXME{{\`E} coerente con la regola dei segni data sopra?  Sembra di s{\`\i},
  ma allora non c'{\`e} differenza\ldots}
  Note that, although $V^{\lor\lor} \simeq V$, the reverse-order evaluation
  and coevaluation maps $\ev'_V$, $\coev'_V$ differ from
  $\ev_{\rdl{V}}$ and $\coev_{\rdl{V}}$ by a sign, because of the
  non-trivialness of $\tau_{V,\rdl{V}}$; indeed, according to
  \ref{thm:ev-rev},
  \begin{align*}
    \ev'_V = \ev_V \circ \tau_{V,\rdl{V}} &: \rdl{v} \otimes v \mapsto (-1)^{v\rdl{v}}
    \pairing{v}{\rdl{v}},
    \\
    \ev_{\rdl{V}} &: \rdl{v} \otimes v \mapsto \pairing{\rdl{v}}{v},
  \end{align*}
  and similarly $\coev'_V = \pm \coev_{\rdl{V}}$.
\end{remark}

The category $\catVect$ of degree $0$ vector spaces inherits a
structure of a tortile autonomous braided symmetric category from
$\catVect^\setZ$, but the braiding restricts to the trivial one.


\subsection{The category of Bosonic states}
\label{sec:bosonic}

Let $V$ be a $\fk$-vector space equipped with a bilinear
\emph{symmetric} non-degenerate form $b: V\tp{2} \to \fk$. Let
$\btpc{V,b}$ be the full subcategory of $\catVect$ which has tensor
powers $V\tp{p}$ as objects; $\btpc{V,b}$ inherits a balanced
symmetric tensor structure from $\catVect$. For each $V\tp{p} \in
\btpc{V,b}$ define maps
\begin{equation*}
  \ev_{V\tp{p}}: V\tp{p} \otimes V\tp{p} \ni (v_1, \ldots, v_p) \otimes (v'_1, \ldots, v'_p) \mapsto
  b(v_1, v'_1) \cdots b(v_p, v'_p) \in \fk \simeq I.
\end{equation*}
Since $b$ is non-degenerate and $V$ is finite-dimensional, $\ev^\sharp$ is
a linear isomorphism, hence a bijection, so $V\tp{p}$ is a self-dual
object and $\btpc{V,b}$ is an autonomous category.

The category $\btpc{V,b}$ can be used as a base for a graphical
calculus description of Feynman diagrams associated to Gaussian
integrals --- see \cite[sec.\
2.8]{murri-fiorenza;feynman}.\FIXME{Piccolo spazio pubblicit{\`a}\ldots}



%%% Local Variables: 
%%% mode: latex
%%% TeX-master: "index"
%%% x-symbol-8bits: nil
%%% End: 
