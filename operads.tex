\RCSID $Id: operads.tex,v 1.2 2006/05/26 14:53:42 rmurri Exp $
%auto-ignore



\chapter{Operads}
\label{cha:operads}

Fix a differential graded $\k$-algebra $K$; the category of
$K$-dg-modules and their morphisms (not necessarily commuting with the
differential) is Abelian and symmetric monoidal with the tensor product
$\otimes := - \otimes_K -$.  In this section, we mainly follow
\cite{kriz-may;operads} for the definition of an operad.
\begin{definition}
  \label{dfn:operad}
  An operad ${\Oo}$ is a collection of $K$-modules $\{{\Oo}(n)\}_{n\geq0}$
  together with 
  \begin{enumerate}[i)]
  \item a unit map $\eta:K \to {\Oo}(1)$; 
  \item a right action of the symmetric group $\Perm{n}$ on ${\Oo}(n)$,
    for all $n\geq0$;
  \item composition maps ${\Oo}(n)\otimes{\Oo}(k_1)\otimes\cdots\otimes{\Oo}(k_n) \xrightarrow{\gamma_{n;
        k_1, \ldots, k_n}} {\Oo}(k)$ for all $n\geq0$ and $k = \sum k_s$; $\deg \gamma
    = 0$ as a map of graded modules.
  \end{enumerate}
  These data are required to satisfy the following compatibility
  relations.
  \begin{enumerate}
  \item\label{o-ass} The following \emph{associativity} diagrams
    commute, for all $n\geq0$, $k_s \geq0$, $j_{sr}\geq0$:
    \begin{equation}
      {\xymatrix{%
          {\Oo}(n) \otimes \bigl( \bigotimes_{s=1}^n {\Oo}(k_n) \bigr) \otimes \bigl( \bigotimes_{s=1}^n
          \bigotimes_{r=1}^{k_s} {\Oo}(j_{sr}) \bigr) 
          &
          {\Oo}(k) \otimes \bigl(\bigotimes_{s=1}^n \bigotimes_{r=1}^{k_s} {\Oo}(j_{sr}) \bigr)
          &
          k := \sum_s k_s
          \\
          &
          {\Oo}(j)
          \\
          {\Oo}(n) \otimes \biggl( \bigotimes_{s=1}^n \Bigl( {\Oo}(k_s) \otimes \bigl(
          \bigotimes_{r=1}^{k_s} {\Oo}(j_{sr}) \bigr) \Bigr) \biggr)
          &
          {\Oo}(n) \otimes \bigl( \bigotimes_{s=1}^n {\Oo}(j_s) \bigr)
          &
          j_r := \sum_{r=1}^{k_s} j_{sr}
          %%
          \ar@{<->}  "1,1";"3,1" _{\txt{signed \\ reordering}} 
          \ar "1,1";"1,2" ^{\gamma_{n; k_1, \ldots, k_n} \otimes \id}
          \ar "3,1";"3,2" ^{\id \otimes \bigotimes_{s=1}^n \gamma_{k_s; j_{s1}, \ldots,
              j_{sk_s}}} 
          \ar "1,2";"2,2" ^{\gamma_{k; j_{11}, \ldots, j_{nk_n}}}
          \ar "3,2";"2,2" _{\gamma_{n; j_1, \ldots, j_n}}
          }}
      \label{eq:o-ass}
    \end{equation}
    where the ``signed reordering'' acts is the appropriate
    composition of commutators in the symmetric category.
  \item\label{o-unit} The following \emph{unit} diagrams commute.
    \begin{equation}
      {\xymatrix{%
          {\Oo}(n)\otimes K\tp{n}
          \\
          & {\Oo}(n)
          \\
          {\Oo}(n) \otimes {\Oo}(1)\tp{n}
          \ar "1,1";"3,1" _{\id \otimes \eta\tp{n}}
          \ar "1,1";"2,2" 
          \ar "3,1";"2,2" _{\gamma_{n; 1, \ldots, 1}}
          }
        \qquad
        \xymatrix{%
          K \otimes {\Oo}(n)
          \\
          & {\Oo}(n)
          \\
          {\Oo}(1) \otimes {\Oo}(n)
          \ar "1,1";"3,1" _{\eta \otimes \id}
          \ar "1,1";"2,2" 
          \ar "3,1";"2,2" _{\gamma_{1; n}}
          }}
      \label{eq:o-unit}
    \end{equation}
  \item\label{o-perm} The following \emph{equivariance} diagrams
    commute, for $\sigma \in \Perm{n}$ and $\tau_s \in \Perm{k_s}$, the
    permutation $\sigma(k_1, \ldots, k_n) \in \Perm{k}$ permutes blocks of
    lengths $k_1$, \ldots, $k_n$ like $\sigma$ permutes letters; $\tau_1 \oplus \cdots
    \oplus \tau_n$ permutes letters in the $s$-th block as $\tau_s$ does:
    \begin{equation}
      {\xymatrix@C=36pt{%
          {\Oo}(n) \otimes {\Oo}(k_1) \otimes \cdots \otimes {\Oo}(k_n)
          & 
          {\Oo}(n) \otimes {\Oo}(k_{\sigma_1}) \otimes \cdots \otimes {\Oo}(k_{\sigma_n})
          \\
          {\Oo}(k)
          &
          {\Oo}(k)
          \ar "1,1";"1,2" ^{\sigma \otimes T_{\sigma^{-1}}}
          \ar "1,1";"2,1" _{\gamma_{n; k_1, \ldots, k_n}}
          \ar "1,2";"2,2" ^{\gamma_{n; k_{\tau_1}, \ldots, k_{\tau_n}}}
          \ar "2,1";"2,2" ^{\sigma(k_{\sigma_1}, \ldots, k_{\sigma_n})}
          }}
      \label{eq:o-perm1}
    \end{equation}
    and
    \begin{equation}
      {\xymatrix@C=48pt{%
          {\Oo}(n) \otimes {\Oo}(k_1) \otimes \cdots \otimes {\Oo}(k_n)
          & 
          {\Oo}(k)
          \\
          {\Oo}(n) \otimes {\Oo}(k_1) \otimes \cdots \otimes {\Oo}(k_n)
          &
          {\Oo}(k)
          \ar "1,1";"1,2" ^(.70){\gamma_{n; k_1, \ldots, k_n}}
          \ar "2,1";"2,2" _(.70){\gamma_{n; k_1, \ldots, k_n}}
          \ar "1,1";"2,1" _{\id \oplus (\tau_1 \otimes \cdots \otimes \tau_n)}
          \ar "1,2";"2,2" ^{\tau_1 \oplus \cdots \oplus \tau_n}
          }}
      \label{eq:o-perm2}
    \end{equation}
  \end{enumerate}
\end{definition}
Operads may be defined in any symmetric tensor category. For our
purposes, it will suffice to restrict always to the category of
$K$-dg-modules.

The prototypical operad is the \emph{endomorphism operad} $\EndOp[M]$ of some
$K$-dg-module $M$, which is defined by:
\begin{equation*}
  \EndOp[M](n) := \Hom( M\tp{n}, M).
\end{equation*}
The unit is given by the identity map $\id_M \in \Hom(M,M) =
\EndOp[M](1)$; the $\Perm{n}$-action is the (signed) permutation of
tensor product factors, and, finally, the maps $\gamma_{n; k_1, \ldots, k_n}$
are given by 
\begin{equation*}
  {\xymatrix{%
      \Hom(M\tp{n}, M) \otimes \Hom(M\tp{k_1}, M) \otimes \cdots \otimes \Hom(M\tp{k_n},
      M)
      \\
      \Hom(M\tp{n}, M) \otimes \Hom(M\tp{k}, M\tp{n})
      \\
      \Hom(M\tp{k}, M)
      \ar "1,1";"2,1" ^{\id \otimes (\text{$n$-fold tensor product of
          maps})}
      \ar "2,1";"3,1" ^{\circ}
      }}
\end{equation*}
It is trivial to verify that $\EndOp[M]$ is an operad.

A \emph{morphism of operads} $\phi: {\Oo}' \to {\Oo}''$ is a collection of
$K$-dg-modules morphisms $\{\phi_n: {\Oo}'(n) \to {\Oo}''(n)\}$ satisfying the
obvious compatibility conditions coming from diagrams
\eqref{eq:o-ass}--\eqref{eq:o-perm2}. 

\subsection{Signs in operads}
\label{sec:signs-operads}
The ``signed reordering'' of diagram \eqref{eq:o-ass} may appear as to
introduce an unnatural sign: infact it is not so, the reason being
that a Koszul sign is hidden into the $\gamma$'s, as the following example
shows.

Let $M$ be an $K$-dg-module, $A \in \Hom(M\tp{2}, M)$, $B_1 \in
\Hom(M,M)$ and $B_2 \in \Hom(M\tp2, M)$. According to the definition of
operad, $\gamma_{2; 1,2}(A \otimes B_1 \otimes B_2) \in \Hom(M\tp3, M)$, so we pick
$x_1 \otimes x_2 \otimes x_3 \in M\tp3$ and reckon:
\begin{equation*}
  y := (A \otimes B_1 \otimes B_2) (x_1 \otimes x_2 \otimes x_3) = A \otimes \bigl( (B_1 \otimes B_2)
  (x_1 \otimes x_2 \otimes x_3) \bigr),
\end{equation*}
so, by the Koszul sign convention,
\begin{multline*}
  y = A \otimes \bigl( (-1)^{x_1 B_2} B_1(x_1) \otimes B_2(x_2 \otimes x_3) \bigr) \\ 
  = (-1)^{x_1 B_2} A\bigl( (-1)^{x_1 B_2} B_1(x_1) \otimes B_2(x_2 \otimes x_3)
  \bigr).
\end{multline*}
Therefore,
\begin{equation}
  \label{eq:1}
  \gamma(A \otimes B_1 \otimes B_2) (x_1 \otimes x_2 \otimes x_3) = (-1)^{x_1B_2}  A\bigl( (-1)^{x_1 B_2} B_1(x_1) \otimes B_2(x_2 \otimes x_3) \bigr).
\end{equation}

Let us apply this to a particular case of \eqref{eq:o-ass}: pick $C_1,
C_2, C_3 \in \Hom(M,M)$, then use \eqref{eq:1} to walk \eqref{eq:o-ass}
bottom-up: first on the left-side path,
\begin{multline*}
  A\Bigl( B_1\bigl( C_1(x_1) \bigr), B_2\bigl(C_2(x_2), C_3(x_3)\bigr)
  \Bigr)  \\ = (-1)^{(x_1 + C_1)B_2} \gamma(A \otimes B_1 \otimes B_2) \bigl( C_1(x_1)
  \otimes C_2(x_2) \otimes C_3(x_3) \bigr)  \\ = (-1)^{C_1B_2 + x_1B_2 + x_1(C_2 +
    C_3) + x_2C_3}  \times \\ \gamma\bigl( \gamma(A \otimes B_1 \otimes B_2) \otimes C_1 \otimes C_2 \otimes C_3
  \bigr) (x_1 \otimes x_2 \otimes x_3),
\end{multline*}
then on the right-side path,
\begin{multline*}
  A\Bigl( B_1\bigl( C_1(x_1) \bigr), B_2\bigl(C_2(x_2), C_3(x_3)\bigr)
  \Bigr) \\ = (-1)^{x_2 C_3} A\bigl( \gamma(B_1 \otimes C_1) (x_1), \gamma(B_2 \otimes C_2
  \otimes C_3) (x_2 \otimes x_3) \bigr) \\ = (-1)^{x_2 C_3 + x_1(B_2 + C_2 + C_3)}
  \times \\ \gamma\bigl( A \otimes \gamma(B_1 \otimes C_1) \otimes \gamma(B_2 \otimes C_2 \otimes C_3) \bigr) (x_1 \otimes
  x_2 \otimes x_3).
\end{multline*}
Now we see that the sign $(-1)^{B_2C_1}$ coming from the reordering
\begin{equation*}
T_{(34)}: A \otimes B_1 \otimes B_2 \otimes C_1 \otimes C_2 \otimes C_3 \mapsto (-1)^{B_2 C_1} A \otimes
B_1 \otimes C_1 \otimes B_2 \otimes C_2 \otimes C_3
\end{equation*}
is the one needed to make diagram \eqref{eq:o-ass} commute.


\section{Algebras over operads}
\label{sec:algebras-over-operads}

Let $X$ be an $K$-dg-module, and ${\Oo}$ a fixed operad.
\begin{definition}
  A structure of ${\Oo}$-algebra on $X$ is an operad morphism ${\Oo} \to
  \EndOp[X]$ of ${\Oo}$ into the endomorphism operad of $X$. 
\end{definition}
So, elements in ${\Oo}(n)$ are interpreted as maps $X\tp{n} \to X$, that
is, they are $n$-ary operations on $X$. In the sequel we shall see how
to give a operadic formulation of common algebraic structures.

The following proposition encodes many well-known structure transfer
theorems. 
\begin{theorem}
  \label{prop:structure-transfer}
  Any morphism ${\Oo}\to{\Oo}'$ induces an ${\Oo}$-algebra structure on every
  ${\Oo}'$-algebra $X$, by means of the composition ${\Oo} \to {\Oo}'
  \to\EndOp[X]$.
\end{theorem}

In an equivalent manner, an ${\Oo}$-algebra structure on $X$, is given by
maps
\begin{equation*}
  \phi_n : {\Oo}(n) \otimes X\tp{n} \to X,
\end{equation*}
which satisfy associativity, unit and equivariance conditions coming
from diagrams \eqref{eq:o-ass}--\eqref{eq:o-perm2}, that we can
express by requiring the following diagrams to commute:
\begin{gather}
  \label{eq:o-alg}
  {\xyc\xymatrix{%
      {\Oo}(n) \otimes {\Oo}(k_1) \otimes \cdots \otimes {\Oo}(k_n) \otimes X\tp{k}
      &
      {\Oo}(k) \otimes X\tp{k}
      \\
      &
      X
      \\
      {\Oo}(n) \otimes \bigl({\Oo}(k_1) \otimes X\tp{k_1}) \otimes \cdots \otimes \bigl({\Oo}(k_n) \otimes
      X\tp{k_n})
      &
      {\Oo}(n) \otimes X\tp{n} 
      %%
      \ar@{<->} "1,1";"3,1"  _{\txt{signed \\ reordering}}
      \ar "1,1";"1,2" ^(.65){\gamma \otimes \id}
      \ar "1,2";"2,2" ^{\phi_k}
      \ar "3,2";"2,2" _{\phi_n}
      \ar "3,1";"3,2" _(.75){\id \otimes \phi\tp{k}}
      }\endxyc}
  \\
  {\xyc\xymatrix@C=36pt{%
      {\Oo}(n) \otimes X\tp{n}
      &
      {\Oo}(n) \otimes X\tp{n}
      \\
      {\Oo}(n) \otimes X\tp{n}
      & 
      X
      %%
      \ar "1,1";"1,2" ^{\id \otimes \varepsilon(\sigma)}
      \ar "1,1";"2,1" _{\sigma \otimes \id\tp{n}}
      \ar "1,2";"2,2" ^{\phi_n}
      \ar "2,1";"2,2" _{\phi_n}
      }\endxyc}
  \\
  {\xyc\xymatrix{%
      X
      &
      {\Oo}(1) \otimes X
      &
      X
      %%
      \ar "1,1";"1,2" ^(.35){\eta \otimes \id}
      \ar "1,2";"1,3" ^(.60){\phi_1}
      \ar@/_1pc/ "1,1";"1,3" _{\id_X}
      }\endxyc}
\end{gather}

\begin{example}[Free algebras over an operad] Fix an operad ${\Oo}$, and an
  $K$-dg-module $V$. Define the free ${\Oo}$-algebra $X := F_{\Oo}(V)$ by:
  \begin{equation*}
    X^p := {\Oo}(p) \otimes V.
  \end{equation*}
  In order to define the structure maps $\phi$, observe that 
  \begin{multline*}
    \bigl( {\Oo}(n) \otimes X\tp{n} \bigr)^p = {\Oo}(p) \otimes \bigoplus_{p_1 + \cdots + p_n =
      p} \bigl( ({\Oo}(p_1) \otimes X^{p_1}) \otimes \cdots \otimes ({\Oo}(p_n) \otimes X^{p_n})
    \bigr) 
    \\
    \simeq \bigl( {\Oo}(p) \otimes \bigoplus_{p_1 + \cdots + p_n =
      p} ({\Oo}(p_1) \otimes \cdots \otimes {\Oo}(p_n)) \bigr) \otimes (X\tp{n})^p,
  \end{multline*}
  so a direct sum of the maps $\gamma_{n; p_1, \ldots, p_n}$ will do the job.
\end{example}


\subsection{The operad of associative algebras}
\label{sec:operad-assoc}
If $X$ is an associative algebra, then a map $m: X\tp2 \to X$ is
defined, which satisfies the constraint
\begin{equation}
  \label{eq:ass}
  m(x_1, m(x_2, x3)) = m(m(x_1, x_2), x_3).
\end{equation}
Let us define an operad $\Ao$ which governs associative algebras; we
need an element $\mu \in \Ao(n)$ for every $n$-ary operation on $X$. Now,
the product of two elements $x_1 \otimes x_2 \mapsto m(x_1, x_2) =: x_1x_2$ is a
binary operation, but also the product in reverse order ($x_1 \otimes x_2
\mapsto m(x_2, x_1) =: x_2x_1$) is; moreover, we can form $n$-ary
operations composing product maps (e.g., $x_1 \otimes x_2 \otimes x_3 \otimes x_4 \mapsto
m(m(x_1, x_2), m(x_4, x_3)) = (x_1x_2)(x_3x_4)$).  The associativity
relation \eqref{eq:ass} tells us that any monomial $x_{\sigma_1} x_{\sigma_2}
\cdots x_{\sigma_n}$ defines a $n$-ary operation by composition of binary
products, independently of the way we put parentheses in it.
Therefore, we can state that $\Ao(n)$ is the $K$-module freely
generated by the set of all possible products of symbols $x_1, \ldots,
x_n$ with every $x_j$ appearing once and once only; for example,
\begin{equation*}
  \Ao(1) := K,
  \quad \Ao(2) := \lspan{ x_1x_2, x_2x_1}_K, 
  \quad \Ao(3) := \lspan{ x_{\sigma_1}x_{\sigma_2}x_{\sigma_3} : \sigma \in \Perm{3}}_K.
\end{equation*}
The $\Perm{n}$-action on $\Ao(n)$ consists in permuting the $x_j$'s.
The structure maps $\gamma_{n; k_1, \ldots, k_n}: \Ao(n) \otimes \Ao(k_1) \otimes \cdots \otimes
\Ao(k_n) \to \Ao(k_1 + \cdots + k_n)$ replace the $x_j$ in $\mu \in \Ao(n)$ by a
monomial $\mu_j \in \Ao(k_j)$, simultaneously replacing $x_l$ in $\mu_j$
with $x_{k_1 + \cdots + k_{j-1} + l}$.
  
The operad $\Ao$ is most easily described using a graphical notation.
Depict an element of $\Ao(n)$ as a \emph{binary} tree having $n$ leaves
(inputs) and one root (the output); the leaves are numbered, and
$\Perm{n}$ acts by permuting numbers on leaves. Such trees can be
easily seen to correspond to the meaningful insertion of $n-2$ pairs
of parentheses into a monomial $\mu \in \Ao(n)$.
\begin{center}
  \begin{tabular}{ccc}
    \begin{xytree}
      \branch{%
        \leaf{1}
        \leaf{2}
        }
    \end{xytree}
    &
    \begin{xytree}
      \branch{%
        \leaf{1}
        \branch{%
          \leaf{2}
          \leaf{3}
          }
        }
    \end{xytree}
    &
    \begin{xytree}
      \branch{%
        \branch{%
          \leaf{1}
          \leaf{3}
          }
        \branch{%
          \leaf{2}
          \leaf{4}
          }
        }
    \end{xytree}
    \\
    $x_1x_2$
    &
    $x_1(x_2x_3)$
    &
    $(x_1x_3)(x_2x_4)$
  \end{tabular}
\end{center}
The structure map $\gamma_{n; k_1, \ldots, k_n}$ simply grafts the tree $t_j
\in {\Oo}(k_j)$ onto the $j$-th input of $t \in {\Oo}(n)$ and renumbers the
leaves of $t_j$; for example,
\begin{equation*}
  \gamma \left\lgroup 
    \xytreec
      \branch{%
        \leaf{1}
        \leaf{2}
        }
    \endxytreec
    \otimes
    \xytreec
      \branch{%
        \branch{%
          \leaf{2}
          \leaf{3}
          }
        \leaf{1}
        }
    \endxytreec
    \otimes
    \xytreec
      \branch{%
        \branch{%
          \leaf{1}
          \leaf{3}
          }
        \branch{%
          \leaf{2}
          \leaf{4}
          }
        }
    \endxytreec
  \right\rgroup
  =
  \xytreec
    \branch{%
      \branch{%
        \branch{%
          \leaf{2}
          \leaf{3}
          }
        \leaf{1}
        }
      \branch{%
        \branch{%
          \leaf{4}
          \leaf{6}
          }
        \branch{%
          \leaf{5}
          \leaf{7}
          }
        }
      }
  \endxytreec
\end{equation*}
  
The whole family of numbered binary trees is generated by elements in
$\Ao(2)$ via the composition maps $\gamma$; infact,
\begin{equation*}
  \Ao(2) = \lspan{%
    \xytreec
      \branch{%
        \leaf{1}
        \leaf{2}
        }
    \endxytreec
    ,
    \xytreec 
      \branch{%
        \leaf{2}
        \leaf{1}
        }
    \endxytreec
    }.
\end{equation*}
What is more, the associativity relation \eqref{eq:ass} can be
rewritten as:
\begin{equation}
  \label{eq:ass-tree}
  \xytreec
    \branch{%
      \leaf{1}
      \branch{%
        \leaf{2}
        \leaf{3}
        }
      }
  \endxytreec
  =
  \xytreec
    \branch{%
      \branch{%
        \leaf{1}
        \leaf{2}
        }
      \leaf{2}
      }
  \endxytreec
\end{equation}
So, the operad $\Ao$ is the quotient of the family of binary
trees by relations of the form:
\begin{equation*}
  \gamma \left\lgroup
    T_0 \otimes \left\lgroup
  \xytreec
    \branch{%
      \leaf{1}
      \branch{%
        \leaf{2}
        \leaf{3}
        }
      }
  \endxytreec
  -
  \xytreec
    \branch{%
      \branch{%
        \leaf{1}
        \leaf{2}
        }
      \leaf{3}
      }
  \endxytreec
  \right\rgroup
  \otimes T_1 \otimes T_2 \otimes T_3 \right\rgroup 
\end{equation*}
where $T_0$, $T_1$, $T_2$ and $T_3$ are arbitrary binary trees.  It is
easy to check that these relations imply that any tree in $\Ao$ has
a representative such that left-wing branches have no ramification;
such trees will be called \emph{regular trees} in the sequel.

\begin{definition}
  The space $\Ao(n)$ is the $K$-linear span of the set of regular
  binary trees.  It is a dg-module with the trivial differential $D =
  0$.

  The collection $\{\Ao(n)\}$ forms an operad with the structure maps
  given by the grafting operation $\gamma$ and  the obvious $\Perm{n}$ action.
\end{definition}
It is an easy exercise to check the following.
\begin{theorem}
  Any associative algebra is an algebra over $\Ao$, and vice-versa.
\end{theorem}


\section{Modules over an operad algebra}
\label{sec:modules}
Fix an operad ${\Oo}$ and an ${\Oo}$-algebra $X$, and an $K$-module $M$. 
\begin{definition}
  A structure of $({\Oo}, X)$-module on $M$ is given by a collection of
  maps $\psi_n: {\Oo}(n+1) \otimes X\tp{n} \otimes M \to M$ which satisfy the following
  compatibility relations.
\begin{gather}
  \label{eq:o-2}
  {\xyc\xymatrix{%
      {\Oo}(n) \otimes {\Oo}(k_1) \otimes \cdots \otimes {\Oo}(k_n) \otimes X\tp{k-1} \otimes M & {\Oo}(k) \otimes
      X\tp{k-1} \otimes M
      \\
      {{\Oo}(n) \otimes \bigotimes_{j=1}^{n-1}\bigl({\Oo}(k_j) \otimes X\tp{k_j}) \otimes {\Oo}(k_n) \otimes
        X\tp{k_n-1} \otimes M}
      \\
      {\Oo}(n) \otimes X\tp{n-1} \otimes M & X
      %%
      \ar@{<->} "1,1";"2,1"  _{\txt{signed \\ reordering}}
      \ar "1,1";"1,2" ^(.65){\gamma \otimes \id_X\tp{k-1} \otimes \id_M}
      \ar "1,2";"3,2" ^{\psi_k}
      \ar "3,1";"3,2" _{\psi_n}
      \ar "2,1";"3,1" _(.75){\id \otimes \phi\tp{k-1} \otimes \psi_{k_n-1}}
      }\endxyc}
  \\
  {\xyc\xymatrix@C=48pt{%
      {\Oo}(n) \otimes X\tp{n-1} \otimes M
      &
      {\Oo}(n) \otimes X\tp{n-1} \otimes M
      \\
      {\Oo}(n) \otimes X\tp{n-1} \otimes M
      & 
      X
      %%
      \ar "1,1";"1,2" ^{\id_{\Oo} \otimes \varepsilon(\sigma) \otimes \id_M}
      \ar "1,1";"2,1" _{\sigma \otimes \id_X\tp{n-1} \otimes \id_M}
      \ar "1,2";"2,2" ^{\psi_n}
      \ar "2,1";"2,2" _{\psi_n}
      }\endxyc}
  \\
  {\xyc\xymatrix{%
      M
      &
      {\Oo}(1) \otimes M
      &
      M
      %%
      \ar "1,1";"1,2" ^(.35){\eta \otimes \id_M}
      \ar "1,2";"1,3" ^(.60){\psi_1}
      \ar@/_1pc/ "1,1";"1,3" _{\id_M}
      }\endxyc}
\end{gather}
\end{definition}

It is easy to check that modules over an $\Ao$-algebra are the usual
modules over an associative algebra.

Furthermore, one can define a notion of ``universal enveloping
algebra'' in an operadic context; the category of modules over an
${\Oo}$-algebra $X$ is naturally equivalent to that of modules over
the universal enveloping algebra $U_{\Oo}(X)$. As one would expect,
for $\Ao$ one gets the same standard notions of enveloping algebras.




%%% Local Variables: 
%%% mode: latex
%%% TeX-master: "index"
%%% End: 
