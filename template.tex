%% <<filename>>
%%
%% Report on the fatgraphs making up the Kontsevich graph complex of
%% the moduli space M_${<<g>>,<<n>>}$ of punctured Riemann surfaces.
%%
%% Generated by FatGHoL's "latex" command (version <<version>>)
%% from data in '<<checkpoint_dir>>'.
%%
\documentclass[a4paper]{article}
\raggedbottom

% use Knuth's "concrete" fonts, which blend better with the
% typewriter font used for code listings
\usepackage[T1]{fontenc}
\usepackage[boldsans]{ccfonts}

\usepackage{ifpdf}

\usepackage{amsmath}
\usepackage{amsfonts}
\usepackage{colortbl}
\usepackage{hyperref}
\usepackage{longtable}%
  \setcounter{LTchunksize}{100}%
\usepackage{multicol}
\usepackage{tensor}
\usepackage[usenames,dvipsnames]{xcolor}

\ifpdf 
  % NOTE: Xy-PIC 3.8 is needed for PDF support!
  \usepackage[pdf,color,curve,frame,line,poly]{xy}
\else  
  % fall-back to (x)DVI \specials
  \usepackage[xdvi,color,curve,frame,line,poly]{xy}% 
\fi  
\UseCrayolaColors

  
\newcommand{\corner}[3]{\ensuremath{\tensor[^{#2}]{#1}{^{#3}}}}
\newcommand{\cornerjoin}{\to}
% alternate:
%\newcommand{\corner}[3]{\ensuremath{\stackrel{#2}{-}#1\stackrel{#3}{-}}}
%\newcommand{\cornerjoin}{\kern-0.25ex\to\kern-0.25ex}


%% disable section numbering
\setcounter{secnumdepth}{-1}


\begin{document}

\title{Fatgraphs of $M_{<<g>>,<<n>>}$}
\author{%%
  Automatically generated by FatGHoL <<version>>
  \\
  (See: \url{http://fatghol.googlecode.com/})
  }
\date{ <<today>> }
\hypersetup{%%
  pdftitle={Fatgraphs of $M_{<<g>>,<<n>>}$},%%
  pdfcreator={FatGHoL},%%
  pdfkeywords={{fatgraphs} {M<<g>>,<<n>>}
               {moduli space} {associative operad}},%%
  }
\maketitle


\noindent{}
There are a total of <<total_num_graphs>> undecorated fatgraphs in the
Kontsevich graph complex of $M_{<<g>>,<<n>>}$, originating
<<total_num_marked_graphs>> marked ones.


% only insert 'Fatgraphs with XXX edges' entries
\setcounter{tocdepth}{1}
\tableofcontents

\clearpage


\section{Notation}

We denote $G_{m,j}$ the $j$-th graph in the set of undecorated
fatgraphs with $m$ edges; the symbol $G_{m,j}^{(k)}$ denotes the
$k$-th inequivalent marking of $G_{m,j}$.

Fatgraph vertices are marked with lowercase latin letters
``a'', ``b'', ``c'', etc.; edges are marked with an arabic
numeral starting from ``1''; boundary cycles are denoted
by lowercase greek letters ``$\alpha$'', ``$\beta$'', etc.

Automorphisms are only listed if the automorphism group is
non-trivial.  Automorphisms are specified by their action on the set
of vertices, edges, and boundary cycles: for each automorphism $A_k$,
a table line lists how it permutes vertices, edges and boundary cycles
relative to the identity morphism $A_0$.

If a fatgraph has automorphisms that reverse the orientation of the
associated cell, then its picture is crossed out;
orientation-reversing automorphisms are marked with a ``$\dag$''
symbol in the automorphism table.

If a fatgraph is orientable, a ``Markings'' section lists all the
inequivalent ways of assigning distinct numbers $\{0, \ldots, n-1\}$
to the boundary cycles; this is of course a set of representatives for
the orbits of $\mathfrak{S}_n$ under the action of $\mathrm{Aut}(G)$.

A separate section lists the differential of marked fatgraphs; graphs
with null differential are omitted.  If no marked fatgraph has a
non-zero differential, the entire section is dropped.

Boundary cycles are specified using a ``sequence of corners''
notation: each corner is represented as \corner{L}{p}{q} where $L$ is
a latin letter indicating a vertex, and $p$,~$q$ are the attachment
indices of the incoming and outgoing edges, respectively.  Attachment
indices match the Python representation of the vertex: e.g., if
a$=$\verb'Vertex([0,0,1])', the two legs of edge~$0$ have attachment
indices~0 and~1, and the boundary cycle enclosed by them is
represented by the (single) corner~\corner{a}{0}{1}.

\clearpage


<<for section in sections>>
\section{<<section.title>>}
<<section.intro>>

There are <<section.num_graphs>> undecorated fatgraphs in this section,
originating <<section.num_marked_graphs>> marked fatgraphs.

<<for G in section.graphs >>
\vspace{2em}
\subsection{The Fatgraph $<<G.name>>$%
  <<if not G.orientable>>{\em (non-orientable)}<<endif>>}
\vspace{-1em}
\begin{tabular}{lr}
  \begin{minipage}{0.5\textwidth}
  {% left column: Xy-Pic diagram
<<G.latex_repr>>
  }%
  \end{minipage}
  &% right column: Python code
  \begin{minipage}{0.4\textwidth}
\begin{verbatim}
<<G.python_repr>>
\end{verbatim}
  \end{minipage}
\end{tabular}
\vspace{-3em}

\subsubsection{Boundary cycles}
<<G.boundary_cycles>>

<<if 'automorphisms' in G>>
\subsubsection{Automorphisms}
<<G.automorphisms>>
<<endif>>

<<if 'markings' in G>>
\subsubsection{Markings}
<<G.markings>>
<<endif>>

<<if 'differentials' in G>>
\subsubsection{Differentials}
<<G.differentials>>
<<endif>>

<<endfor>>%% end graph


<<endfor>>%% end sections


\end{document}

%%% Local Variables: 
%%% mode: latex
%%% TeX-master: t
%%% End: 
