% auto-ignore


\chapter{Kontsevich' Model of 2D Quantum Gravity}
\label{cha:kontsevich}


In the famous paper \cite{witten;intersection-theory}, E.~Witten
related two models of two-dimensional quantum gravity: one based on
matrix integrals and random surfaces, and one that used supersymmetry
to reduce the action integral to a computation of intersection numbers
on the Deligne-Mumford moduli space of stable curves. The partition
function in the first theory was known to obey the KdV differential
equation, so, the belief that both approaches were good models of the
same phenomena led Witten to conjecture that the same equation would
also hold for the partition function in the second theory, namely, for
the generating series of intersection numbers of the moduli space of
curves.

In this form, the conjecture appeared as an infinite hierarchy of
recursion relations among these intersection numbers; Witten himself
proved the first cases using techniques of algebraic geometry.
Surprisingly, Kontsevich' proof
\cite{kontsevich;intersection-theory;1992} of the full conjecture took
an entirely different route, steering towards Feynman diagrams and a
combinatorial description of the moduli space; what is more, it
provided a model for 2D quantum gravity, which had not been devised
before!  The later paper \cite{witten;kontsevich-model} established
that all three models of 2D quantum gravity are equivalent.

This chapter is an exposition of Kontsevich' Theorem \cite[Theorem
1.1]{kontsevich;intersection-theory;1992} relating the generating
series of intersection numbers on the moduli space of stable curves
with the matrix integral that now bears his name, as it provides a
template for treating intersection theory on $A_\infty$-classes.  The
exposition will focus on those techniques that are needed in the
sequel and most suitable to generalization.

This is by no means a complete overview; I may only refer the
interested reader to the survey papers by Looijenga
\cite{looijenga;intersection-theory} and Dijkgraaf
\cite{dijkgraaf;intersection-theory}, which are very readable yet
complete introductions to the results obtained in this field.


\section{Witten's conjecture and Kontsevich' theorem}
\label{sec:witten-classes}

\begin{definition}
  The line bundle $\Lb_i \to \Mbar_{g,n}$ is defined by the requirement
  that the fiber over $(C; x_1, \ldots, x_n)$ be the cotangent space
  $T^*_{x_i} C$.
\end{definition}
These $\Lb_i$'s are bundles in a orbifold sense: their first Chern
classes are rational cohomology classes.
\begin{definition}
  The Miller classes are the rational cohomology classes $\psi_i :=
  c_1({\Lb}_i)$.
\end{definition}

For non-negative integers $\nu_1, \ldots, \nu_n$, define
\begin{equation}
  \label{eq:intersection-index}
  \langle \tau_{\nu_1} \cdots \tau_{\nu_n} \rangle_{g,n} := \int_{\Mbar_{g,n}} \psi_1^{\nu_1} \cdots \psi_n^{\nu_n}.
\end{equation}
The integral on right-hand side can be nonzero only if $\nu_1 + \cdots + \nu_n
= 3g - 3 + n$; notice that $g$ is uniquely determined, given $n$ and
$\nu_1$, \ldots, $\nu_n$. Therefore, we drop the subscript $g,n$ in the
equation above, since no confusion can occur. Mumford
\cite{mumford;enumerative-geometry} remarked that, by a theorem of
Arakelov,
\begin{equation}
  \label{eq:positive-intersection}
  \langle\tau_{\nu_1} \cdots \tau_{\nu_n}\rangle \geq 0.
\end{equation}
Thus the permutation group $\Perm{n}$ acts trivially on
$\Htop(\Mbar_{g,n})$ by \eqref{eq:positive-intersection}, so we can
introduce the abridged notation:
\begin{equation}
  \label{eq:intersection-index-cpt}
  \langle\tau_0^{d_0} \tau_1^{d_1} \cdots \tau_k^{d_k} \rangle := \langle \underbrace{\tau_0 \cdots
    \tau_0}_{\text{$d_0$ times}} \cdots \underbrace{\tau_k \cdots
    \tau_k}_{\text{$d_k$ times}} \rangle.
\end{equation}

\begin{definition}\label{dfn:F-and-Z}
  The ``free energy'' is the following formal series in an infinite
  number of indeterminates:
  \begin{equation}\label{eq:F-dfn}
    F(t_0, t_1, \ldots) := \sum_{g, n, \nu_1, \ldots, \nu_n} \langle
    \tau_{\nu_1} \cdots \tau_{\nu_n} \rangle t_{\nu_1} \cdots t_{\nu_n}.
  \end{equation}
  
  The ``partition function'' is the formal series 
  \begin{equation}\label{eq:Z-dfn}
    Z(t_*) := \exp F(t_*).
  \end{equation}
\end{definition}
With the notation of \eqref{eq:intersection-index-cpt}, we may
rewrite:
\begin{equation}\label{eq:F-dfn-cpt}
  F(t_*) = \sum_{g, d_1, \ldots, d_k} \langle \tau_0^{d_0} \cdots \tau_k^{d_k} \rangle \frac
  {t_0^{d_0} \cdots t_k^{d_k}} {d_0! \cdots d_k!} =: \sum_{g, d_*} \langle \tau_*^{d_*}
  \rangle t_*^{d_*}/d_*!,
\end{equation}
a formula that will occasionally come handy.

Witten \cite{witten;intersection-theory} conjectured that $Z$ satisfies
a certain hierarchy of differential equations (the KdV hierarchy).
Kontsevich \cite{kontsevich;intersection-theory;1992} expressed $Z$ as
an asymptotic expansion of a certain matrix integral; this is enough
to prove Witten's conjecture
(\cite{kontsevich;intersection-theory;1992},
\cite{witten;kontsevich-model}). Let us state this assertion
precisely.

Let $\Hn := \Hermitian[N]$ be the space of all $N \times N$ Hermitian
matrices; let $\Lambda \in \Hn$ be a positive definite diagonal
Hermitian matrix, that is,
\begin{equation*}
  \Lambda = \text{diag}( \Lambda_1, \ldots, \Lambda_N ),
\end{equation*}
for real numbers $\Lambda_i > 0$. The \emph{Miwa coordinates} are the
functions defined by:
\begin{equation*}
  t_i (\Lambda) := -(2i-1)!! \tr \Lambda^{-2i-1}.
\end{equation*}
For any given $N$, coordinates $t_1$, \ldots, $t_N$ are independent on
$\Hermitian[N]/U(N)$. Now, $\Hn$ is a finite-dimensional real vector
space, therefore, there is a naturally defined translation-invariant
measure $\ud X$. Let $\ud\mu_\Lambda(X)$ be the Gaussian measure associated
to $\exp ( -1/2 \tr \Lambda X^2)$, that is,
\begin{equation*}
  \ud \mu_\Lambda(X) := c^{-1} \cdot \exp ( -1/2 \tr \Lambda X^2)
  \ud X, \qquad c := \int_\Hn  \exp ( -1/2 \tr \Lambda X^2) \ud X.
\end{equation*}

\begin{definition}\FIXME{Togliere? Direi che si pu\`o dare per
    scontato che chi legge conosca la definizione di espansione asintotica\ldots}
  A formal series $\sum a_j t^j$ is an asymptotic expansion of a real
  function of a real variable $f(t)$ at a point $t_0$ iff
  \begin{equation*}
    \lim_{t \to t_0} t^{-(k+1)} \bigl( f(t) - {\textstyle\sum_0^k a_j t^j}
    \bigr) = a_{k+1}.
  \end{equation*}
\end{definition}
If an asymptotic expansion exists, it is unique. The definition
generalizes easily to functions of many variables; however, the
asymptotic expansion of a function depends on its domain, that is, $f$
and its restriction to a smaller domain might have not the same
asymptotic expansion at the same point $t_0$.
\begin{theorem}[Kontsevich]
  \label{thm:kontsevich}
  The formal series $Z(t_0(\Lambda), t_1(\Lambda), \ldots)$ is an asymptotic
  expansion of
  \begin{equation*}
    \label{eq:K}
    \int_{\Hermitian[N]} \exp(\I/6 \cdot \tr X^3) \ud\mu_\Lambda(X),
  \end{equation*}
  when $\Lambda^{-1} \to 0$ and $N \gg 0$.
\end{theorem}
Therefore, one can compute $F(t_0(\Lambda), t_1(\Lambda), \ldots)$ through 
\begin{equation*}
  F(t_0(\Lambda), t_1(\Lambda), \ldots) \asymp \log     \int_\Hn \exp({\I}/{6} \tr X^3) \ud\mu_\Lambda(X).
\end{equation*}
The intersection numbers of $\Mbar_{g,n}$ can be computed by choosing
$N > 3g - 3 + n$; in the limit $N \to \infty$ we retrieve information on
all moduli spaces. 

The proof of Theorem~\ref{thm:kontsevich} is broken up into a series of
lemmas that will be proved in the rest of this chapter.


\section{Kontsevich' calculations}
\label{sec:calculation}

Kontsevich' praised calculation provides the bridge between the
combinatorial approach to the geometry of moduli spaces and the
Feynman diagrams theory of matrix integrals. In view of later
applications \cite{witten;kontsevich-model}, this should be
regarded as the very important result in Kontsevich' proof of Witten's
conjecture.


\subsection{Combinatorial expression of Witten's classes}
\label{sec:wittens-classes-comb}
\FIXME{Tutta questa sezione pu{\`o} ridursi al solo Theorem~\ref{thm:comb-bundle}?}
Let $C_N$ be the cyclic group of order $N$. Define $B_N := \setR^N / C_N$
to be the orbicell of sequences $[l_1, \ldots, l_N]$ of non-negative real
numbers modulo cyclic permutations, endowed with the quotient
topology. There exist $N+1$ natural inclusion maps $\iota_j: B_N \ni [l_1,
\ldots, l_N] \mapsto [l_1, \ldots, l_{j-1}, 0, l_j, \ldots, l_N] \in B_{N+1}$; these
make $\{B_N, \iota\}$ into an inductive family of topological spaces.
Finally, define
\begin{equation*}
  B := \varinjlim B_N.
\end{equation*}
Every point of $B$ has a unique representative $[l_1, \ldots, l_N]$
enjoying $l_j > 0$ for all $j$. It is convenient to introduce
``barycentric'' coordinates $[p; s_1, \ldots, s_N]$ given by $p := \sum l_j$
and $s_j := l_j/p$.

Define $U_N := \{ (p; \Theta) | p>0, \ \Theta\subseteq S^1, \card{\Theta} = N\}$; give it the
subspace topology inherited from $(S^1)^N$. The group $S^1$ acts on
every $U_N$, and there are surjections $f_\sigma$, for $\sigma \in C_N$:
\begin{multline*}
  E_N := \{ (p; \theta_1, \ldots, \theta_N) | 0\leq \theta_1 < \ldots < \theta_N < 1\} \ni (p;
  \theta_1, \ldots, \theta_N) 
  \\
  \stackrel{f_\sigma}{\longmapsto} [p; \E^{\I\theta_{\sigma_1}}, \ldots,
  \E^{\I\theta_{\sigma_N}}] \in U_N.
\end{multline*}
Put $U_{\leq N} := \bigcup_{j \leq N} U_j$; the inclusions $U_{\leq N} \hookrightarrow U_{\leq
  N+1}$ define an inductive family. Finally, define:
\begin{equation*}
  E := \varinjlim U_{\leq N}.
\end{equation*}
The coordinate patch $E^\sigma_N := (E_N, f_\sigma)$ covers the stratum $U_N
\subseteq E$.

Define projection maps $E^\sigma_N \to B_N$ by 
\begin{equation*}
  E_N^\sigma \ni (p; \theta_1, \ldots, \theta_N) \mapsto [p; \theta_2 - \theta_1, \theta_3 - \theta_2, \ldots, 1
  + \theta_1 - \theta_N] \in B_N;
\end{equation*}
they can be glued into a continuous surjection $\ell: E \to B$, whose
fibers are the orbits of the action of $S^1$ on $E$. It is easy to
check that the spaces $B$ and $E$ are contractible and that $E$ is an
$S^1$-bundle on $B$.
\begin{lemma}
  \label{thm:comb-bundle}
  There exists a differential $1$-form $\alpha$ on $E$ such that:
  \begin{enumerate}
  \item on any patch $E^\sigma_N$, $\alpha$ is given by
    \begin{equation*}
      \alpha := \sum_{j=1}^N s_j \ud\theta_j, \qquad s_j= \theta_{j+1} - \theta_j, \ s_N = 1 +
      \theta_1 - \theta_N;
    \end{equation*}
  \item $\alpha$  restricts to the angle form on the fibers of $\ell$.
  \item   the differential $\ud\alpha$ relates to the first Chern class of $E$
    according to
    \begin{equation*}
      \ud\alpha = -s^*c_1(E);
    \end{equation*}
  \item the differential $\ud\alpha$ is the pull-back on $E$ of a $1$-form $\omega$
    on $B$, defined by:
    \begin{equation*}
      \omega = \sum_{1\leq j\leq k\leq N-1} \ud s_j \land \ud s_k.
    \end{equation*}
  \end{enumerate}
\end{lemma}
\begin{proof}
  Items 1)--2) may be proved by direct calculation; 3) is a standard
  result: consult, for instance, \cite{bott-tu}; finally 4) can be
  seen by induction on $N$.
\end{proof}

Now, the turnkey observation is that points in $B$ may be regarded as
circles with $N$ points marked, that is, ribbon graphs with only
2-valent vertices; so, we can define maps $b_j: \Mcomb_{g,n} \to B$ which
send the $j$-th hole $\beta$ (made of edges $\alpha_1$, \ldots, $\alpha_N$) to the
point $[\ell_{\alpha_1}, \ldots, \ell_{\alpha_N}] \in B_N \subseteq B$. 
\begin{theorem}[Kontsevich, {\cite[Theorem
    2.3]{kontsevich;intersection-theory;1992}}] The map $\M_{g,n} \times
  \setR^n \to B$, which is the composition of the maps $b_j$ with the
  isomorphism $\M_{g,n} \times \setR^n \simeq \Mcomb_{g,n}$, extends continuously to
  $\Mbar_{g,n} \times \setR^n$. The inverse image of the bundle $E \to B$ under
  $b_j$ is naturally isomorphic to the $S^1$-bundle associated with
  the dual of the complex line bundle ${\Lb}_j$. The pull-back $\omega_j$ of
  the form $\omega$ (defined in Theorem~\ref{thm:comb-bundle}) under $b_j$
  represents the class $c_1({\Lb}_j)$.
\end{theorem}


\subsection{Kontsevich' Main Identity}
\label{sec:KMI}

Results from the previous section allow us to write
\begin{equation}
  \label{eq:kontsevich-6}
  \langle \tau_{\nu_1} \cdots \tau_{\nu_n} \rangle = \int_{\pi^{-1}(p^\circ_*)} \omega_1^{\nu_1} \land \cdots
  \land \omega_j^{\nu_n},
\end{equation}
but actually, as Kontsevich proved, one can draw a much stronger
conclusion. 

\begin{theorem}[Kontsevich' Main Identity]
  Let $\RG_{g,n}$ be the set of isomorphism classes of numbered
  ribbon graphs with genus $g$ and $n$ boundary components. If
  $\alpha$ is an edge of a graph $\Gamma \in \RG_{g,n}$, let $\alpha^+$
  and $\alpha^-$ be the \emph{numbers} assigned to the two holes on
  the sides of $\alpha$. For real positive indeterminates $\lambda_1$,
  \ldots, $\lambda_N$, the following relation holds:
  \begin{multline*}\label{eq:KMI}
    \sum_\nu \langle \tau_{\nu_1} \cdots \tau_{\nu_n} \rangle \prod_{j=1}^n \frac {(2\nu_j - 1)!!}
    {\lambda_j^{2\nu_j + 1}} = \sum_{\Gamma \in \RG_{g,n}} \frac
    {2^{-\card{\Vertices{\Gamma}}}} {\card{\Aut \Gamma}} \prod_{\alpha \in \Edges{\Gamma}}
    \frac {2} {\lambda_{\alpha^+} + \lambda_{\alpha^-}}, 
    \\ \sum \nu = 3g - 3 + n.
  \end{multline*}
\end{theorem}
\begin{proof}
  Define a differential $2$-form on $\Mbarcomb_{g,n}$ by $\Omega := \sum
  p_j^2 \omega_j$. Recall that the metric data $(\ell_\alpha)_{\alpha \in \Edges{\Gamma}}$
  define local coordinates on the orbicell $M(\Gamma)$; it is easy to check
  that the restriction of $\Omega$ to the fibers of $\pi$ is non-degenerate
  and has constant coefficients in the coordinates $\ell_\alpha$. 
  \FIXME{Omettere la dimostrazione, visto che \`e nell'articolo di
    Kontsevich?}

  Therefore, $\Omega^q$, with $q := 3g -3 + n = \dim \M_{g,n}$, is a
  volume form on the fibers of $\pi$.  The product $\Omega^q/q! \times
  dp_*$ is a volume form on the whole $\Mcomb_{g,n}$ and one can check
  that the ratio of measures
  \begin{equation*}
    \rho := (\Omega^q/q! \times {dp_1 \land \cdots \land dp_n}) / (d\ell_1 \land \cdots \land d\ell_N) 
  \end{equation*}
  is a \emph{constant}, independent of $\Gamma$ (that is, local expression
  in a cell $M(\Gamma)$), and depending only on $g$ and $n$; more precisely,
  \begin{equation*}
    \rho = 2^{2n + 5g - 5} = 2^{q - \card{\Vertices{\Gamma}} + \card{\Edges{\Gamma}}}.
  \end{equation*}
  The proof of this last fact is quite technical and intricate: see
  \cite[Appendix C]{kontsevich;intersection-theory;1992}.

  Consider the integral
  \begin{equation*}
    U := \int_{\Mbarcomb_{g,n}} \Omega^q/q! \cdot \exp(-{\textstyle\sum} \lambda_j p_j) dp_*.
  \end{equation*}
  On the one hand, from formula
  \begin{equation*}
    \int_{\pi^{-1}(p^\circ_*)} \Omega^q/q! = (1/q!) \int_{\pi^{-1}(p^\circ_*)} \bigl( p_1^2
    c_1({\Lb}_1) + \cdots + p_n^2 c_1({\Lb}_n) \bigr)^q,
  \end{equation*}
  and \eqref{eq:kontsevich-6}, one can compute:
  \begin{equation}
    \label{eq:8}
    U = \sum_\nu \langle \tau_{\nu_1} \cdots \tau_{\nu_n} \rangle \prod_{j=1}^n (2\nu_j)!!
    \lambda_j^{-2\nu_j -1}, \qquad \sum \nu_j = q.
  \end{equation}

  On the other hand, since the ratio $\rho$ is constant on all of
  $\Mcomb_{g,n}$, then 
  \begin{equation*}
    U = \rho \cdot \int_{\Mcomb_{g,n}} \exp\bigl(-{\textstyle\sum} \lambda_j p_j \bigr) 
    \abs{d\ell_1 \land \cdots \land d\ell_N}.
  \end{equation*}
  Since the Strebel triangulation of $\Mcomb_{g,n}$ is indexed by
  ribbon graphs, we can immediately say that the above integral is a
  sum of terms corresponding to ribbon graphs; what is more, since the
  integral can be computed by restricting to an open stratum, we may
  limit the sum to trivalent ribbon graphs only. A little
  combinatorics shows that, for a fixed trivalent ribbon graph
  $\Gamma$,
  \begin{equation*}
    \exp( -{\textstyle \sum} \lambda_j p_j) = \prod_{\alpha \in \Edges{\Gamma}} \exp -\ell_\alpha(\lambda_{\alpha^+} +
    \lambda_{\alpha^-}),
  \end{equation*}
  so that we can finally compute:
  \begin{equation}
    \label{eq:7}
    U = \sum_{\Gamma \in \RG_{g,n}} \frac {1} {\card{\Aut \Gamma}} \prod_{\alpha \in
      \Edges{\Gamma}} \frac{1} {\lambda_{\alpha^+} +
      \lambda_{\alpha^-}}.
  \end{equation}

  Multiply by $2^{-q}$ and equate \eqref{eq:7} and \eqref{eq:8} to get
  the thesis.
\end{proof}

Take a formal sum, over all $g,n$, of the LHS of the Main Identity
\eqref{eq:KMI}, and substitute $\Lambda_{j_k}$ for $\lambda_k$:
\begin{multline}
  \label{eq:KMI-F}
  F(t_0(\Lambda), t_1(\Lambda), \ldots) = \sum_{n, \nu_1, \ldots, \nu_n} 1/n! \langle \tau_{\nu_1}
  \cdots \tau_{\nu_n} \rangle t_{\nu_1}(\Lambda) \cdots t_{\nu_n}(\Lambda)
  \\
  = \sum_{n, \nu_1, \ldots, \nu_n} (-1)^n/n! \langle \tau_{\nu_1} \cdots \tau_{\nu_n} \rangle
  \sum_{1 \leq j_1, \ldots, j_n \leq N} \prod_{k=1}^n (2\nu_k -1)!! \Lambda_{j_k}^{-2\nu_k
    - 1}
  \\
  = \sum_{\substack{\Gamma \in \Strebel[g,n] \\ c: \Holes{\Gamma} \to \{1,
      \ldots, N\}}} \frac {(\I/2)^{\card{\Vertices{\Gamma}}}} {\card{\Aut
      \Gamma}} \prod_{\alpha \in \Edges{\Gamma}} \frac{2} {\lambda_{c(\alpha^+)} +
    \lambda_{c(\alpha^-)}}.
\end{multline}
So we have expressed $F(t_*(\Lambda))$ as a sum of analytic expressions
computed from graphs; the right-hand side comes out to be a Feynman
diagrams expansion of a matrix integral.


\section{The Kontsevich model}
\label{sec:matrix-models}
\everyxy={0,<2em,0em>:,(0,0.5),} % scala per i diagrammi

The Kontsevich matrix model for 2D quantum gravity, first defined in
\cite{kontsevich;intersection-theory;1992}, provides a bridge from
equation~\eqref{eq:KMI-F} to Hermitian matrix integrals. The
Kontsevich model embodies the ``standard matrix model'' as a
particular case.

A cyclic algebra structure will be introduced on the vector space
$\Hermitian[N]$ of $N\times N$ Hermitian matrices, so that we can apply the
general results of Chapter~\ref{cha:fd} on Feynman diagrams.

Let $V$ be an $N$-dimensional Hilbert space. The space $\Ends(V)$
has a natural Hermitian inner product
\begin{equation*} 
  \inner{X}{Y}\joinrel:=\tr(X^*Y),
\end{equation*}
which induces the standard Euclidean inner product $\inner{X}{Y} =
\tr(XY)$ on the real subspace
\begin{equation*}
  \Hermitian[N]\joinrel:=\{X\in \Ends(V) | X^*=X\}
\end{equation*}
of Hermitian operators.  

For any positive definite Hermitian operator $\Lambda$, we can define
a new Euclidean inner product on $\Hermitian[N]$ by
\begin{equation*}
  \inner[\Lambda]{X}{Y} \joinrel:= 
  \onehalf \left(\tr(X\Lambda Y) + \tr(Y\Lambda X)\right).
\end{equation*}

Now, define cyclic tensors
\begin{equation*}
  T_k:\Hermitian[N]\tp{k} \ni X_1 \otimes X_2 \otimes \dots \otimes X_k 
  \mapsto 
  \tr(X_1X_2\cdots X_k) \in \setC;
\end{equation*}
These $T_k$, together with the inner product $\inner[\Lambda]{-}{-}$, define
a cyclic algebra structure on $\Hermitian[N] \otimes \setC \cong \Ends(V)$, that is
called the ``Kontsevich model''.  A graphical calculus $Z$ for the
Kontsevich model is defined on the ribbon category $\RG$ of ribbon
graphs.

\begin{lemma}\label{thm:KMI-Z}
  The following formula holds:
  \begin{equation*}
    \label{eq:KM}
    Z (\Gamma) =
    \sum_{c} \prod_{\ell\in
      \Edges{\Gamma}} \frac{2}{\Lambda_{c(\ell^+)} + \Lambda_{c(\ell^-)}},
    \qquad c\colon\Holes{\Gamma} \to \{1, \dots, N\}
  \end{equation*}
  where:
  \begin{itemize}
  \item $\Lambda_1, \dots, \Lambda_N$ are the eigenvalues of $\Lambda$;
  \item $c$ runs over all colorings of holes of $\Gamma$ in $N$ colors;
  \item $\ell^+$, $\ell^-$ are the two holes bounded by the edges $\ell$ (they
    are not necessarily distinct).
  \end{itemize}
\end{lemma}
\begin{proof}
  To evaluate $Z(\Gamma)$ we need an explicit expression for the
  Casimir element $\gamma = \coev_{\Hermitian[N],\Lambda}(1)$ of the cyclic
  algebra $(\Hermitian[N]_\setC, \inner[\Lambda]{-}{-}, T_1, T_2,
  \dots)$.  Since $\Lambda$ is Hermitian positive definite, there
  exists an orthonormal basis $\{e_i\}$ of $V$ in which
  \begin{equation*}
    \Lambda=\diag(\Lambda_1,\Lambda_2\dots,\Lambda_N),
  \end{equation*}
  for some $\Lambda_i$ positive real numbers. Any choice of a like basis
  induces an identification of $V$ with $\setC^N$, and, consequently,
  of $\Ends(V)$ with the space $\setC^{N\times N}$ of $N \times N$ complex
  matrices. Let $\{E_{ij}\}$ be the canonical basis for $\setC^{N\times N}$:
  \begin{equation*}
    (E_{ij})_{kl}=\delta_{ik}\delta_{jl}. 
  \end{equation*}
  A basis for $\Hermitian[N]$ is given by matrices
  \begin{equation*}
    e_{ij}=\begin{cases}
      \frac{1}{\sqrt{2}} (E_{ij}+E_{ji}) &\text{if $i<j$,}\\
      E_{ii}                             &\text{if $i=j$,}\\
      \sqrt{-\onehalf} (E_{ij}-E_{ji})   &\text{if $i>j$.} 
    \end{cases}
  \end{equation*}
  It is orthonormal with respect to the inner product
  $\inner{-}{-}$, whereas 
  \begin{equation*}
    \inner[\Lambda]{e_{ij}}{e_{kl}}
    = \frac{\Lambda_i + \Lambda_j}{2} \delta_{ij,kl},
  \end{equation*}
  i.e., the matrix of $\inner[\Lambda]{-}{-}$ with respect to the
  basis $\{e_{ij}\}$ is
  \begin{equation*}
    g = \diag\left(\left\{ \frac{\Lambda_i + \Lambda_j}{2}
      \right\}\right). 
  \end{equation*}
  So we get the following expression for the Casimir element:
  \begin{equation*}
    \gamma = \sum_{i,j} \frac{2}{\Lambda_i+\Lambda_j} e_{ij} \otimes e_{ij}. 
  \end{equation*}
  Rewrite this identity as:
  \begin{align*}
    \gamma &=\sum_{i}\frac{1}{{\Lambda_i}}e_{ii}\otimes
    e_{ii}+\sum_{i<j}\frac{2}{{\Lambda_i+\Lambda_j}}e_{ij}\otimes
    e_{ij}+\sum_{i>j}\frac{2}{{\Lambda_i+\Lambda_j}}e_{ij}\otimes
    e_{ij} 
    \\
    &=\sum_{i}\frac{1}{{\Lambda_i}}e_{ii}\otimes
    e_{ii}+\sum_{i<j}\frac{2}{{\Lambda_i+\Lambda_j}}(e_{ij}\otimes
    e_{ij}+e_{ji}\otimes
    e_{ji}),
  \end{align*}
  but, for $i<j$,
  \begin{align*}
    e_{ij}\otimes e_{ij} &+ e_{ji}\otimes e_{ji} = \\
    &\qquad + \frac{1}{
      2}(E_{ij}\otimes E_{ij} + E_{ij}\otimes E_{ji} + E_{ji}\otimes
    E_{ij}+E_{ji}\otimes E_{ji})\\
    &\qquad - \frac{1}{ 2}(E_{ij}\otimes E_{ij} - E_{ij}\otimes E_{ji} -
    E_{ji}\otimes
    E_{ij}+E_{ji}\otimes E_{ji})\\
    &= E_{ij}\otimes E_{ji} + E_{ji}\otimes E_{ij}. 
  \end{align*}
  So,
  \begin{multline}\label{eq:casimir}
    \gamma = \sum_{i}\frac{1}{{\Lambda_i}}E_{ii}\otimes
    E_{ii}+\sum_{i<j}\frac{2}{{\Lambda_i+\Lambda_j}}(E_{ij}\otimes E_{ji} + E_{ji}\otimes E_{ij})\\
    = \sum_{i,j}\frac{2}{{\Lambda_i+\Lambda_j}}E_{ij}\otimes E_{ji}.
  \end{multline}

  According to the rules of graphical calculus, evaluation
  $Z(\Gamma)$ is performed through the correspondence
  \begin{equation*}
    {\xy*!LC\xybox{\rgvertex7\loose1\loose2\missing3%
        \loose4\loose5\missing6\loose7}\endxy}
    \leftrightarrow
    T_k,
    \qquad
    {\xy\vloop-\endxy}
    \leftrightarrow
    \gamma=\coev_{\Hermitian[N],\Lambda}(1). 
  \end{equation*}
  If we introduce the notation
  \begin{equation*}
    {\xy\vloop-,(0.05,0.5)*\txt{${}_i\
        {}_j$},(1,0.5)*\txt{${}_j\
        {}_i$}\endxy}=\frac{2}{\Lambda_i+\Lambda_j} E_{ij}\otimes E_{ji},
  \end{equation*}
  then we can depict \eqref{eq:casimir} as
  \begin{equation*}
    {\xy\vloop-\endxy}
    = \sum_{i,j}
    {\xy\vloop-,(0.05,0.5)*\txt{${}_i\
        {}_j$},(1,0.5)*\txt{${}_j\ {}_i$}\endxy},
  \end{equation*}
  which turns $Z(\Gamma)$ into a sum of ribbon graphs equipped with a
  number in $\{1, \dots, N\}$ on each side of every edge, and
  operators $T_k$ on each $k$-valent vertex. 
  
  Since
  \begin{equation}\label{eq:vertices}
    T_k(E_{i_1j_1}\otimes E_{i_2j_2}\otimes\cdots\otimes
    E_{i_{k}j_k})=\delta_{j_1i_2}\delta_{j_2i_3}\cdots
    \delta_{j_{k-1}i_k}\delta_{j_ki_1},
  \end{equation}
  the only graphs that give non-zero contribution to the sum giving
  $Z(\Gamma)$ are the ones whose boundary components have the same index on
  all edges --- that is, we need only account for graphs equipped with
  a map $c\colon\Holes{\Gamma} \to \{1, \dots, N\}$.  An edge whose sides are
  indexed by $i,j$ brings in a factor $2/(\Lambda_i+\Lambda_j)$; combining this
  with \eqref{eq:vertices}, we can conclude the proof.
\end{proof}

\begin{proof}[Proof of Theorem~\ref{thm:kontsevich}]
  Asymptotic expansions commute with the integral sign, therefore
  \begin{equation*}
    \int_{\Hn} \exp (\I/6 \cdot \tr X^3) \ud\mu_\Lambda(X) \asymp \sum_m
    \frac{\I^m}{2^mm!} \correlator{(\tr
      X^3/3)^m}. 
  \end{equation*}
  
  By Theorem~\ref{thm:FRT}, the $m$-th summand in the right-hand side
  can be expanded into a sum over (possibly non-connected) colored
  ribbon graphs with $m$ trivalent vertices. As $m$ ranges over
  non-negative integers, we get the exponential of the right-hand side
  of \eqref{eq:KMI-F}, which completely proves Theorem~\ref{thm:kontsevich}. 
\end{proof}

\begin{example}[The standard matrix model]
  Take $\Lambda = I$; formula \eqref{eq:KM} specializes to 
  \begin{equation*}
    Z(\Gamma) = \sum_{c} \prod_{\ell\in \Edges{\Gamma}} \frac{2}{\Lambda_{c(\ell^+)} +
      \Lambda_{c(\ell^-)}}
    = \sum_c 1 = N^{\card{\Holes{\Gamma}}}. 
  \end{equation*}
  Therefore, according to Theorem~\ref{thm:FRT},
  \begin{equation}
    \int_{\Hermitian[N]} \exp \left\{ \frac{1}{\hbar}\sum_{j=1}^{\infty}
      \frac{\tr X^j}{j}
    \right\} \ud\mu_I(X) = 
    \sum_\Gamma \frac{N^{\card{\Holes{\Gamma}}}} {\card{\Aut
        \Gamma}}\hbar^{-\card{\Vertices{\Gamma}}}. 
  \end{equation}
  This is known as the ``standard matrix model'' in physics
  literature. 
\end{example}



\everyxy={/r24pt/:} % riportiamo la scala per i diagrammi

%%% Local Variables: 
%%% mode: latex
%%% TeX-master: "index"
%%% End: 
