\RCSID $Id: fd.tex,v 1.3 2006/05/12 11:00:59 rmurri Exp $
%auto-ignore


\chapter{Gaussian integrals and Feynman diagrams}
\label{sec:fd}
\everyxy={0,<2em,0em>:,(0,0.5),} % scala per i diagrammi

This chapter is devoted to showing how a Gaussian integral can be
expanded into a sum of Feynman diagrams, within the framework of the
graphical calculus on symmetric categories.  Depending on the nature
of the integral, this formula will hold either as a strict equality,
or in the sense of asymptotic expansions. In particle physics,
Gaussian integrals and their Feynman diagrams expansions are used to
describe Bosonic statistics.

The first description of Feynman diagrams via Graphical Calculus is
apparently due to R.~Oeckl \cite{oeckl;braided-qft}; the exposition
here, however, comes from joint work with D.~Fiorenza
\cite{murri-fiorenza;feynman}. 


\section{Gaussian measures and the Wick's lemma} 

Let $V$ be a finite dimensional Euclidean space,
with inner product $\inner{-}{-}$. If $\{e_i\}$ is a basis of
$V$, we denote the \emph{coordinate} maps relative to this basis as
$e^i\colon V\to\setR$, and write $v^i$ for the pairing
$\pairing{e^i}{v}$.  The matrix associated to $\inner{-}{-}$ with
respect to the basis $\{e_i\}$ is given by
\begin{equation*}
  g_{ij} \joinrel:= \inner{e_i}{e_j}.
\end{equation*}
As customary, we set $g^{ij} \joinrel:= (g^{-1})_{ij} = \inner{e^i}{e^j}$.

Let now $\ud v$ be a (non trivial) translation invariant measure on
$V$. The function $\E^{-\onehalf\inner{v}{v}}$ is positive and
integrable with respect to $\ud v$.
\begin{definition}\label{dfn:gaussian-measure}
  The probability measure on $V$ defined by
  \begin{equation*}
    \ud\mu(v) \joinrel:= \frac{1}{C} \E^{-\onehalf\inner{v}{v}} \ud v, 
    \qquad
    C := {\int_V \E^{-\onehalf\inner{v}{v}} \ud v},
  \end{equation*}
  is called the \emph{Gaussian measure} on $V$.
\end{definition}
Since a non-trivial translation invariant measure on $V$ is unique up
to a scalar factor, $\ud\mu$ is actually independent of the chosen
$\ud v$.

The symbol $\avg{f}$ denotes the \emph{average} of a function $f$
with respect to the Gaussian measure, i.e.,
\begin{equation*}
  \avg{f} \joinrel:= \int_V f(v)\ud\mu(v).
\end{equation*}

\begin{lemma}[Wick]
  The polynomial functions of the coordinates $v^i$ are integrable
  with respect to $\ud\mu$ and:
  \begin{align}
    \tag{W1}
    \avg{v^{i_1}v^{i_2}\cdots
      v^{i_{2n+1}}} &= 0,
    \\
    \tag{W2}
    \avg{v^{i}v^{j}} &= g^{ij},
    \\
    \tag{W3}
    \avg{v^{i_1}v^{i_2}\cdots
      v^{i_{2n}}} &= \sum\limits_{s\in P}
    g^{i_{s_1} i_{s_2}} g^{i_{s_3} i_{s_4}} \cdots
    g^{i_{s_{2n-1}} i_{s_{2n}}}, 
  \end{align}
  where the sum ranges over all distinct pairings of the set of
  indices $\{i_1,\dots,i_{2n}\}$, i.e., over the set of all partitions
  $\bigl\{\{i_{s_1},i_{s_2}\},\{i_{s_3},i_{s_4}\},\dots\bigr\}$ of
  $\{i_1,i_2,\dots,i_{2n}\}$ into 2-element subsets.
\end{lemma}
For a proof of Wick's lemma see, for instance,
\cite{bessis-itzykson-zuber;graphical-enumeration}.

The inner product $\inner{-}{-}$ extends uniquely to a Hermitian
product on the complex vector space $\cplx{V} \joinrel:= V \otimes \setC$.
Identify $V$ with the subspace $V \otimes \{1\}$ of real vectors in
$\cplx{V}$; $\{ e_i \}$ is a real basis for the complex vector
space $\cplx{V}$. Extend $\avg{-}$ to tensor powers of real
vectors by
\begin{equation*}
  \label{eq:avg-x}
  \avg{v\tp{k}} \joinrel:= \sum \avg{v^{i_1} \cdot \cdots \cdot v^{i_k}}
  e_{i_1} \otimes \cdots \otimes e_{i_k},
\end{equation*}
so Wick's lemma can be recast this way:
\begin{align}
  \label{eq:W1}\tag{W1'}
  \avg{v\tp{ 2n+1}} &= 0,
  \\
  \label{eq:W2}\tag{W2'}
  \avg{v\tp{ 2}} &= \sum_{i,j} g^{ij} e_i
  \otimes e_j,
  \\
  \label{eq:W3}\tag{W3'}
  \avg{v\tp{2n}} &= \sum\limits_{i_1, \dots,
    i_{2n}} \sum\limits_{s\in P} g^{i_{s_1} i_{s_2}} \cdots
  g^{i_{s_{2n-1}} i_{s_{2n}}} e_{i_1} \otimes e_{i_2} \otimes \cdots
  \otimes e_{i_{2n-1}} \otimes e_{i_{2n}},
\end{align}
where the last sum ranges over all distinct pairings of indices in the
set $\{i_1, \dots, i_{2n}\}$.


\section{From Wick's lemma to Feynman diagrams}
\label{sec:wick-to-fd}

The right-hand side of (\ref{eq:W2}) is the Casimir element of
$\{\cplx{V}, \inner{-}{-}\}$; in Reshetikhin-Turaev's graphical
notation, we can rewrite \eqref{eq:W2} as
\begin{equation*}
  \avg{v\otimes v}= {\xy\vloop-\endxy}\ .
\end{equation*}
The graphical notation becomes particularly suggestive 
(and useful) when applied to \eqref{eq:W3}:
\begin{equation}
  \label{eq:avg-to-casimir}
  \begin{split}
    \avg{v\tp{ 4}} &=
    {%
      {\xy\vloop-,(1.2,0.5),\vloop-\endxy} +
      {\xy\vloop-|(.7)\knothole,(0.5,0.5),\vloop-\endxy} + 
      {\xy\vloop-,(0.7,0):(0.2,0.7),\vloop-\endxy}
    }\quad, \\
    \ldots& \\
    \avg{v\tp{ 2n}} &=
    {%
      \left({\xy\vloop-,(1.2,0.5),\vloop-,(1.2,0)\endxy} \cdots
        {\xy\vloop-\endxy}\right) +
      \dots +
      {\xy\vloop-,(0.7,0):(0.2,0.7),%
        \vloop-,(0.7,0.2)*\txt{\vdots},(0.24,0):(2.4,2.9)\vloop-\endxy}
      }\quad,
  \end{split}
\end{equation}
the last sum ranging over all possible configurations of $n$ Casimir
elements and the braiding being the trivial one: $x\otimes y\mapsto
y\otimes x$.  

In addition, assume we have, for any $k$, a cyclically invariant
$k$-tensor
\begin{equation*}
  T_{k}:\cplx{V}{}\tp{k} \to \setC;
\end{equation*}
which has the graphical representation
\begin{equation*}
  \underbrace{\xy(0.3333,0):
    (0,0)="a",
    (2,0)="b",
    (4,0)="c",
    (6,0)="d",
    (3,4)="v",
    (3,5)="w",
    "v";"a"**\dir{-},
    "v";"b"**\dir{-},
    "v";"d"**\dir{-},
    "c"*\txt{\ldots},
    "w"*\txt{$T_{k}$},
    "v"*{\bullet}
    \endxy}_{k}
\end{equation*}

The data $(V, \inner{-}{-}, T_1, T_2, \dots)$ define a cyclic
algebra structure, so we can use Reshetikhin-Turaev's graphical
calculus $\Gamma\mapsto Z(\Gamma)$ to compute averages
\begin{equation*}
  \bigl\langle T_1(v)^{l_1} T_2(v\tp{ 2})^{l_2} \cdots
  T_{k}(v\tp{ k})^{l_k} \bigr\rangle.
\end{equation*}
\begin{lemma}\label{thm:feynman-avg-gc}
  Any average $\bigl\langle T_1(v)^{l_1} T_2(v\tp{ 2})^{l_2}
  \cdots T_{k}(v\tp{ k})^{l_k} \bigr\rangle$ is a linear
  combination $\sum \alpha_\Gamma Z(\Gamma)$ where $\Gamma$ runs
  in the set of ribbon graphs of type (0,0) with $l_i$ vertices of
valence $i$, for $i=1,\dots,k$.
\end{lemma}
\begin{proof}
  By linearity of the $T_k$'s and \eqref{eq:avg-x},
  \begin{equation*}
    \avg{T_1(v)^{l_1} \cdots T_{k}(v\tp{ k})^{l_k}} = 
    \bigl(T_1\tp{l_1} \otimes \cdots \otimes T_{k}\tp{l_k} \bigr)
    \avg{v\tp{\sum_i il_i}}.
  \end{equation*}
  If $\sum il_i$ is odd, then $\avg{v\tp{\sum il_i}}$ is zero, and the set
  of graphs considered in the statement is empty. If $\sum il_i$ is
  even then, according to the rules of graphical calculus, $\bigotimes_{i=1}^k
  T_i\tp{l_i}$ corresponds to the juxtaposition of $l_1$ univalent
  vertices, $l_2$ bivalent vertices, etc., up to $l_k$ vertices of
  valence $k$. In this case, \eqref{eq:avg-to-casimir} translates
  $\avg{v\tp{\sum il_i}}$ into edges connecting these vertices in all
  possible ways.
\end{proof}

\begin{example}
  For example,
  \begin{align*}
    \avg{\left(T_2(v\tp{ 2})\right)^2} &= 
    {%
      {\xy*!UC\xybox{
          /r 1cm/:
          ,(1,1.2);(1,1.2)**\crv{(0,0)&(2,0)},(1,1.2)*{\bullet}
          ,(1.8,1.2);(1.8,1.2)**\crv{(0.8,0)&(2.8,0)},(1.8,1.2)*{\bullet}
          }\endxy}
      + 2 \cdot 
      {\xy*!UC\xybox{
          /r 1cm/:
          ,(1,1.2);(1.8,1.2)**\crv{(0,0)&(2.8,0)},(1,1.2)*{\bullet}
          ,(1,1.2);(1.8,1.2)**\crv{(1.2,0.5)&(1.6,0.5)},(1.8,1.2)*{\bullet}
          }\endxy}
      }
    \\
    \avg{T_4(v\tp{ 4})} &=
    {%
      {\xy*!UC\xybox{
          /r 1cm/:
          ,(1,1.2);(1,1.2)**\crv{(-0.2,0)&(1.6,0)},(1,1.2)*{\bullet}
          ,(1,1.2);(1.05,.35)**\crv{(1,1.2)&(1.5,0.7)&(1.5,0.2)&(1.05,.35)}
          ,(1,1.2);(0.95,.45)**\crv{(1,1.2)&(0.8,0.5)&(0.95,.45)}
          }\endxy}
      + 2\cdot
      {\xy*!UC\xybox{
          /r 1cm/:
          ,(1,1.2);(1,1.2)**\crv{(-.4,0)&(.95,0)}
          ,(1,1.2);(1,1.2)**\crv{(1.05,0)&(2.4,0)}
          ,(1,1.2)*{\bullet}
          }\endxy}
      }
  \end{align*}
\end{example}

\begin{lemma}\label{thm:feynman-avg-coeff}
  The coefficient $\alpha_\Gamma$ appearing in
  \csref{thm:feynman-avg-gc} is an integer given by:
  \begin{equation*}
    \alpha_\Gamma =
    \frac{1^{l_1} l_1! 2^{l_2} l_2! \cdots k^{l_k} l_k!}
    {\card{\Aut\Gamma}}.
  \end{equation*}
\end{lemma}
\begin{proof}
  Let $X$ be the set of all ribbon graphs obtained by:
  \begin{inparaenum}
  \item juxtaposing $l_1$ vertices of valence $1$, $l_2$
    vertices of valence $2$, etc., up to $l_k$ vertices of valence
    $k$, and,
  \item connecting them in all possible ways by means of arcs.
  \end{inparaenum}
  The constant $\alpha_\Gamma$ counts the number of occurrences of graphs
  isomorphic to $\Gamma$ in the set $X$.
  
  The semi-direct product $K = \prod_{i=1}^k(\Perm{l_i} \rtimes
(\setZ/i\setZ)^{l_i})$
  acts on $X$ as follows: the image of a graph
  $\Phi$ is obtained by permuting vertices of the same valence and
  rotating edges incident to each vertex. Since this action is
  transitive on isomorphism classes,
  \begin{equation*}
    \alpha_\Gamma = \frac {\card{K}} {\card{\Stab_K(\Gamma)}}
    = \frac {\card{K}} {\card{\Aut\Gamma}}
    = \frac{1^{l_1} l_1! 2^{l_2} l_2! \cdots
      k^{l_k} l_k!}{\card{\Aut\Gamma}},
  \end{equation*}
  where $\Stab_K(\Gamma)$ is the stabilizer of $\Gamma$ under the
  action of $K$.
\end{proof}

\begin{theorem}[Feynman-Reshetikhin-Turaev]\label{thm:FRT}
  Let $Z_{x_*}$ be the graphical calculus for the cyclic algebra
  $A(x_*) \joinrel:= (V_{\setC}, \inner{-}{-}, x_1T_1, x_2T_2, \dots)$, where
  $x_1,x_2,\dots$ are complex variables. Then, the following
  asymptotic expansion holds:
  \begin{equation}
    \label{eq:FRT1}
    Z(x_*)\joinrel:=\int_V \exp \biggl\{ \sum_{k=1}^\infty x_k
    \frac{T_k(v\tp{k})}{k}
    \biggr\} \ud\mu(v) 
    = \sum_{\Gamma\in\RG(0,0)} \frac {Z_{x_*}(\Gamma)} {\card{\Aut\Gamma}},
  \end{equation}
  where the sum on the right ranges over the set $\RG(0,0)$ of all
  isomorphism classes of (possibly disconnected) ribbon graphs of type
  (0,0). The formal series $Z(x_*)$ is called the \emph{partition
    function} of the algebra $A(x_*)$.
\end{theorem}
\begin{proof}
  Expand in Taylor series the left-hand side:
  \begin{align*}
    \int_V \exp \biggl\{\sum_{k=1}^\infty &\frac{x_k}{k}T_k(v\tp{k})\biggr\}
    \ud\mu(v)=
    \\
    &= \sum_{n=0}^\infty \sum_{\nu_1, \dots, \nu_n} \frac{x_{\nu_1} \dots
      x_{\nu_n}} {n!\nu_1 \cdots \nu_n} \avg{T_{\nu_1}(v\tp{ \nu_1}) \cdots
      T_{\nu_n}(v\tp{\nu_n})}
    \\
    &= \sum_{k=0}^\infty \sum_{l_1, \dots, l_k} \frac{x_1^{l_1} \cdots x_k^{l_k}}
    {1^{l_1} l_1! 2^{l_2} l_2! \cdots k^{l_k} l_k!}  \avg{T_{1}(v)^{l_1}
      \cdots T_{k}(v\tp{k})^{l_k}}
    \\
    &= \sum_{k=0}^\infty \sum_{l_1, \dots, l_k} \frac
    {\avg{\bigl(x_1T_{1}(v)\bigr)^{l_1} \cdots
        \bigl(x_kT_{k}(v\tp{k})\bigr)^{l_k}}} {1^{l_1} l_1! 2^{l_2}
      l_2! \cdots k^{l_k} l_k!}, 
    \\
    &= \sum_{\Gamma \in \RG(0,0)} \frac{Z_{x_*} (\Gamma)}
{\card{\Aut\Gamma}},
  \end{align*}
  by lemmas~\ref{thm:feynman-avg-gc} and~\ref{thm:feynman-avg-coeff}.
\end{proof}
A similar argument yields:
\begin{equation}
  \label{eq:FRT2}
  \int_V \frac{T_1(v)^{l_1} T_2(v\tp{2})^{l_2} \cdots
    T_k(v\tp{k})^{l_k}} {1^{l_1} l_1! 2^{l_2} l_2! \cdots k^{l_k}
    l_k!} \exp \left\{ \sum_{k=1}^\infty x_k\frac{T_k(v\tp{k})}{k}
    \right\} \ud\mu(v)
    = \sum_{\Gamma} \frac{Z_{x_*}(\Gamma)}{\card{\Aut\Gamma}}
\end{equation}
where the sum in the right-hand side ranges over all ribbon graphs
having $l_i$ ``special'' $i$-valent vertices (for $i=1\dots,k$), and
$\Aut(\Gamma)$ is the group of automorphisms that map the set of special
vertices to itself. In \eqref{eq:FRT2}, the graphical calculus
$Z_{x_*}$ interprets each $i$-valent special vertex as the operator
$T_i$ and each ordinary $i$-valent vertex as the operator $x_iT_i$.
This is the same as considering graphs with two sorts of vertices (see
\csref{rem:many-sorted-graphs}), one decorated by operators $T_i$
(call them ``special vertices''), and the other decorated by $x_iT_i$
(call them ``ordinary'').

%\begin{rem} Notice that we can recast the cyclical invariance of the
%  tensors $T_k$ by saying $\Aut(T_k)=\setZ/k\setZ$. The statement of
%  \csref{thm:FRT} can be rewritten as
%\begin{equation}\label{eq:FRTeqbis}
%    \int_V \exp \biggl\{ \sum_{k=1}^\infty x_k
%\frac{T_k(v\tp{k})}{\card{\Aut(T_k)}}
%    \biggr\} \ud\mu(v)
%    = \sum_{\Gamma} \frac {Z_{x_*}(\Gamma)} {\card{\Aut\Gamma}},
%\end{equation}
%while equation \eqref{eq:FRT2} becomes
%\begin{equation}
%  \label{eq:feynmanturaevduebis}
%  \int_V \frac{T_1(v)^{l_1} T_2(v\tp{2})^{l_2} \cdots
%    T_k(v\tp{k})^{l_k}} {\card{\Aut(T_1^{\otimes l_1}\otimes
%T_2^{\otimes l_2}
%\cdots \otimes T_k^{\otimes l_k})}} \exp \left\{ \sum_{k=1}^\infty
%x_k\frac{T_k(v\tp{k})}{\card{\Aut(T_k)}}
%    \right\} \ud\mu(v)
%    = \sum_{\Gamma} \frac{Z_{x_*}(\Gamma)}{\card{\Aut\Gamma}},
%\end{equation}
%where the last sum ranges over graphs having $l_1$ distinguished
%$1$-valent vertices, $l_2$ $2$-valent ones, etc.
%\end{rem}

\csref{thm:FRT} can be straightforwardly adapted to a symmetric
algebra $(V, \inner{-}{-},\break S_1, S_2, \dots)$.
\begin{theorem}[Feynman-Reshetikhin-Turaev]\label{thm:FRTsym}
  Let $Z_{x_*}$ be the graphical calculus for the symmetric algebra
  $(V_{\setC}, \inner{-}{-}, x_1S_1, x_2S_2, \dots)$, where
$x_1,x_2,\dots$ are
  complex variables. The following asymptotic expansion holds:
  \begin{equation*}
    Z(x_*)\joinrel:=\int_V \exp \biggl\{ \sum_{k=1}^\infty x_k
    \frac{S_k(v\tp{k})}{k!}
    \biggr\} \ud\mu(v)
    = \sum_{\Gamma\in\SG(0,0)} \frac {Z_{x_*}(\Gamma)} {\card{\Aut\Gamma}},
  \end{equation*}
  where the sum on the right ranges over all isomorphism classes of
(possibly disconnected) ordinary graphs of type (0,0).
\end{theorem}

\begin{example} 
  Let $\phi$ be an analytic function defined in the whole $V$. The
  Taylor expansion formula
\begin{equation*}
\phi(v)=\phi(0)+\sum_{n=1}^{\infty}\frac{D^n\phi\vert_0(v^{\otimes
n})}{n!},
\end{equation*}
together with \csref{thm:FRTsym}  gives
\begin{equation}
  \int_V \E^{\phi(v)} \ud
  \mu(v) = \E^{\phi(0)} \cdot \left(\sum_\Gamma
    \left.\frac{D_\Gamma(\phi)}{\card{\Aut\Gamma}}\right\vert_0\right).
\end{equation}
\end{example}

%\begin{rem} Since the tensors $S_k$ are symmetric,
%  $\Aut(S_k)=\Perm{k}$, so we can write the analogues of formulas
%  \eqref{eq:FRTeqbis} and \eqref{eq:feynmanturaevduebis}:
%  \begin{gather}
%    \label{eq:FRTsymeqbis}
%    \int_V \exp \biggl\{ \sum_{k=1}^\infty x_k
%    \frac{S_k(v\tp{k})}{\card{\Aut(S_k)}}   
%    \biggr\} \ud\mu(v)
%    = \sum_{\Gamma} \frac {Z_{x_*}(\Gamma)} {\card{\Aut\Gamma}},
%    \\
%    \label{eq:feynmanturaevsymbis}
%        \int_V \frac{S_1(v)^{l_1} 
%%          S_2(v\tp{2})^{l_2} 
%          \cdots
%          S_k(v\tp{k})^{l_k}} {\card{\Aut(S_1\tp{l_1}\otimes 
%%            S_2\tp{l_2}
%            \cdots \otimes S_k\tp{l_k})}}
%        \exp \left\{ 
%          \sum_{k=1}^\infty x_k 
%          \frac{S_k(v\tp{k})}{\card{\Aut(S_k)}}
%        \right\} \ud \mu(v)
%        = \sum_{\Gamma} \frac{Z_{x_*}(\Gamma)}{\card{\Aut\Gamma}},
%  \end{gather}
%  where, as usual, the last sum ranges over all graphs with $l_1$
%  distinguished $1$-valent vertices, $l_2$ distinguished $2$-valent
%  vertices, etc., up to $l_k$ distinguished $k$-valent vertices. These
%  are, in fact, the usual ``Feynman rules'' found in QED
%  textbooks.
%\end{rem}

\begin{remark}[Generalized Gaussian integrals]
  Denote by $Q(v)$ the Gaussian weight $\E^{-1/2(v,v)}$, and let
  $g^\sharp\colon V\to V^\lor$ be the isomorphism induced by the
  non-degenerate pairing $g\colon V\otimes V\to\fk$. Moreover, let
  $m$ be the usual multiplication on the space $K \joinrel:=
  \fk[[V^\lor]]$ of formal power series on $V$.  In
  \cite{oeckl;braided-qft}, Robert Oeckl shows that the graphical
  version of Wick's lemma is a formal consequence of the following
  properties:
\begin{enumerate}[(G1)]
\item\label{G1} a ``braided Leibniz rule'' for derivations:
\[\partial_w\circ m=m\bigl((\partial_w\otimes
\Id)+\tau^{-1}_{K, K}\circ(\partial_w\otimes\Id) \circ \tau^{}_{K,
  K}\bigr),\] for any $w\in V$ and any $\phi, \psi\in K$;
\item\label{G2} $\partial_w Q = -g^\sharp(w)\cdot Q$, for all $w\in V$;
\item\label{G3} $g^\sharp$ is an isomorphism and
$\ev_V\circ(\Id_V\otimes g^\#)=\ev_V\circ(\Id_V\otimes
g^\#)\circ\tau_{V,V}$;
\item\label{G4} 
%there exists an integral $\displaystyle{\int (\phi\cdot
%    Q)\ud v}$ such that 
$\displaystyle{\int\partial_w(\phi\cdot Q)\ud
v=0}$, for
  any $w\in V$ and any polynomial $\phi\in K$;
\end{enumerate}
Equations (G\ref{G1}-G\ref{G4}) can be used to \emph{define} Gaussian
integrals in the context of \emph{arbirtary} braided monoidal categories.
This has been done by R.~Oeckl, with the development of ``Braided
QFT'' \cite{oeckl;braided-qft}; since equation
\eqref{eq:avg-to-casimir} is a formal consequence of   
(G\ref{G1}-G\ref{G4}), the whole machinery of Feynman diagrams expansions
will be available for these generalized Gaussian integrals too.
A remarkable by-product of this general
theory is that the Berezin super-integrals of fermionic statistics are
simply obtained as ``braided
Gaussian integrals'' for a vector space $V$ endowed with the non-trivial
braiding $x\otimes y\mapsto -y\otimes x$ (the pairing $g$ must,
consequently, be antisymmetric). Therefore, in the particular case
of the symmetric category of super vector spaces (with the usual graded
symmetric braiding), ``braided Gaussian integrals'' provide an unified
language for both statistics encountered in standard quantum field theory:
bosons correspond to even vectors and fermions correspond to odd vectors
\cite[Sections~3.3 and~3.4]{oeckl;spin-and-statistics}.
\end{remark}
%Note that (G\ref{G1}) and (G\ref{G2}) hold for any pairing $g$;
%(G\ref{G3}) is equivalent to $g$ being non-degenerate; 
%(G\ref{G4}) holds if and only if $g$ is positive
%definite.  
%
%Therefore, given an arbitrary non-degenerate bilinear form $g$, the
%whole machinery of Wick's lemma (and graph expansions, consequently)
%will be available if only we can define an ``integral'' $\int(\ -\ )
%\ud v$ according to (G\ref{G4}).  The most important example of such
%``generalized Gaussian integrals'' is the Berezin super-integral.
%
%These issues have been addressed by R. Oeckl in the setting of a  
%general braided monoidal category, with the development of ``Braided
%QFT'' \cite{oeckl;braided-qft}; in the particular case of the symmetric
%category of super vector spaces (with the usual graded symmetric
%braiding), one gets a unified language for both 
%statistics encountered in standard quantum field theory: bosons
%correspond to even vectors and fermions correspond to odd vectors
%\cite[Sections~3.3 and~3.4]{oeckl;spin-and-statistics}.
%\end{remark}

%\begin{example}
%  \label{xmp:getzler-formula}
%  We want to show how this machinery can be used to derive a
%  graph expansion formula cited in the introduction to
%  \cite{getzler-kapranov}. If we are given several symmetric tensors
%  $\{S_{k,\alpha_k}\}_{\alpha_k\in I_k}$ for any index $k$, the formula in
%  \csref{thm:FRTsym} generalizes to
%  \begin{equation}
%    \label{eq:FRTeqter}
%    \int_V \exp \biggl\{ \sum_{k=1}^\infty\sum_{\alpha_k\in I_k}
%x_k{\nobreakspace}y_{\alpha_k}
%    \frac{S_{k,\alpha_k}(v\tp{k})}{k!}   
%    \biggr\} \ud\mu(v)
%    = \sum_{\Gamma \in \SG^I(0,0)} \frac {Z_{x_*,y_*}(\Gamma)}
%{\card{\Aut\Gamma}}.
%\end{equation}
%where $\SG^I(0,0)$ denotes the set of isomorphism classes of (possibly
%disconnected) graphs whose $k$-valent vertices are decorated by
%elements of the set $I_k$. 

%In particular, if we set, for any $k$,
%\begin{equation*}
%I_k=\mathbb{N}, \quad
%x_k=1, \quad
%y_g=\hbar^g, \quad
%S_{k,g}=0\quad \text{ if } \quad 3g-3+k\leq0,
%\end{equation*}
%and rescale the inner product on $V$ by a factor $1/\hbar$ (i.e., we take
%$(1/\hbar)\inner{-}{-}$ instead), then:
%\begin{equation}
%    \int_V \exp \biggl\{\frac{1}{\hbar} \sum_{k=1}^\infty\sum_{g=0}^\infty
%\hbar^g
%\frac{S_{k,g}(v\tp{k})}{k!}
%    \biggr\} \ud\mu_\hbar(v)
%    = \sum_{g=0}^\infty\hbar^g\sum_{\Gamma \in \textrm{M}(g,0)}
%\hbar^{-b_0(\Gamma)}\frac
%{Z(\Gamma)}
%{\card{\Aut\Gamma}}.   
%\end{equation}
%where $\textrm{M}(g,0)$ denotes the set of isomorphism classes of
%(possibly non connected) genus $g$ modular graphs with no legs. The
%\emph{genus} of the modular graph $\Gamma$ is the integer $g(\Gamma)\joinrel:=\sum_v
%g(v)+\dim H^1(\Gamma)$ and $b_0(\Gamma) = \dim H_0(\Gamma)$.  

%A similar formula holds for modular graphs with legs. A leg can be
%considered as an edge ending in a univalent vertex of a distinguished
%kind, so, fix a linear operator $\zeta \colon V \to \setC$ and extend graphical
%calculus so to evaluate univalent vertices to $\zeta$:
%\begin{equation}\label{eq:modulargraphs}
%    \int_V \exp \biggl\{\frac{1}{\hbar}
%\left(\zeta(v)+\sum_{k=1}^\infty\sum_{g=0}^\infty
%\hbar^g
%\frac{S_{k,g}(v\tp{k})}{k!}
%    \right)\biggr\} \ud\mu_\hbar(v)
%    = \sum_{g,n=0}^\infty \! \sum_{\Gamma \in \textrm{M}(g,n)}
%    \hbar^{g-b_0(\Gamma)} \frac{Z(\Gamma)} {\card{\Aut\Gamma}}.
%\end{equation}
%\end{example}


\section{Application: asymptotic expansion of the ``free energy'' functional}
The logarithm of the partition function $Z(x_*)$ is called the
\emph{free energy} of the cyclic (resp.\ symmetric) algebra, and is
denoted by $F(x_*)$; it admits a graph expansion, too. However,
expansion of the partition function is a sum over \emph{all} graphs,
whereas expansion of the free energy involves only \emph{connected}
ones.

\begin{lemma}\label{thm:feynman-of-Z}
  The free energy $F(x_*)\joinrel:= \log Z(x_*)$ admits a
  Feynman-Reshetikhin-Turaev expansion in ribbon (resp. ordinary)
  graphs, given by:
  \begin{equation}
    \label{eq:feynman-of-Z}
    F(x_*)
    = \sum_{\Gamma\text{\rmfamily\upshape\ connected}}
    \frac{Z_{x_*}(\Gamma)}{\card{\Aut\Gamma}}.
  \end{equation} 
\end{lemma}
\begin{proof}
  Exponentiate
  \begin{equation*}
    \sum_{\Gamma\textrm{ connected}}
    \frac{Z_{x_*}(\Gamma)}{\card{\Aut\Gamma}}
  \end{equation*}
  to find:
  \begin{equation*}
    \exp \left\{\sum_{\Gamma}
      \frac {Z_{x_*} (\Gamma)} {\card{\Aut\Gamma}}\right\}
    = \sum_{k=0}^\infty \oneof{k!} \sum_{\Gamma_1, \dots,
      \Gamma_k} \frac {Z_{x_*}(\Gamma_1) \cdots
      Z_{x_*}(\Gamma_k)} {\card{\Aut\Gamma_1} \cdots
      \card{\Aut\Gamma_k}},
  \end{equation*}
  where each $\Gamma_i$ is a connected graph.  Now recall that
  juxtaposition defines a tensor product $\otimes$ in the category of
  graphs (cf.~\csref{dfn:graph-category}) and that $Z_{x_*}$ is
  multiplicative with respect to this structure:
    \begin{equation*}
      Z_{x_*} (\Gamma_1 \otimes \dots \otimes \Gamma_k)
      = Z_{x_*} (\Gamma_1) \otimes \dots \otimes Z_{x_*}
      (\Gamma_k).
    \end{equation*}
    Therefore,
    \begin{equation*} 
      \exp\left\{ \sum_{\Gamma\textrm{ connected}}
        \frac {Z_{x_*} (\Gamma)} {\card{\Aut\Gamma}}\right\} 
      = \sum_{k=0}^\infty 
      \sum_{\substack{\Gamma_1, \dots, \Gamma_k \\ \text{connected}}}
      \frac {Z_{x_*}(\Gamma_1 \otimes \cdots
        \otimes \Gamma_k)} {k! \card{\Aut\Gamma_1} \cdots
        \card{\Aut\Gamma_k}}
    \end{equation*}
    For a graph $\Phi$ having $k$ connected components $\Gamma_1, \dots,
    \Gamma_k$.  Let $I_\Phi$ be the set of all possible juxtapositions of
    $\Gamma_1, \ldots, \Gamma_k$; all graphs in $I_\Phi$ are isomorphic to $\Phi$. The
    semi-direct product $K$ of $\Perm{k}$ and $\Aut\Gamma_1 \times \dots \times
    \Aut\Gamma_k$ acts transitively on $I$; the stabilizer of any element
    is isomorphic to $\Aut\Phi$. Therefore,
    \begin{equation*}
      \card{I_\Phi} = \frac {\card{K}} {\card{\Stab\Phi}}
      = \frac {k! \card{\Aut\Gamma_1} \cdots \card{\Aut\Gamma_k}}
      {\card{\Aut\Phi}}.
    \end{equation*}
    So we reckon:
    \begin{equation*}
      \exp \left\{\sum_{\Gamma\textrm{ connected}}
        \frac {Z_{x_*} (\Gamma)} {\card{\Aut\Gamma}}\right\} 
      = \sum_{\Phi} \frac{Z_{x_*}(\Phi)}
      {\card{\Aut\Phi}}=Z(x_*).
    \end{equation*}
  \end{proof}
%  For example, taking logarithm of both sides of equation
%  \eqref{eq:modulargraphs} we find
%  \begin{equation}\label{eq:logmodulargraphs}
%    \log\int_V \exp \biggl\{\frac{1}{\hbar}
%    \left(\zeta(v)+\sum_{k=1}^\infty\sum_{g,n}
%      \hbar^g
%      \frac{S_{k,g}(v\tp{k})}{k!} 
%    \right)\biggr\} \ud\mu_\hbar(v)
%    = \frac{1}{\hbar}\sum_{g,n}\hbar^g\!\!\!\!\!\sum_{\Gamma \in
%      \textrm{M}(g,n)\sptext{conn}}
%    \frac{Z(\Gamma)}
%    {\card{\Aut\Gamma}}.
%  \end{equation}
  
\everyxy={/r24pt/:} % riportiamo la scala per i diagrammi

%%% Local Variables: 
%%% mode: latex
%%% TeX-master: "index"
%%% x-symbol-8bits: nil
%%% End: 
